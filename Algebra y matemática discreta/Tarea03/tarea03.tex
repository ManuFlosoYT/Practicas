\documentclass{amsart}
\usepackage[margin = 2cm]{geometry}
\usepackage[utf8]{inputenc}
\usepackage[usefamily=sage]{pythontex} 
\usepackage{colortbl} 
\usepackage{xcolor}

\newtheorem{ejer}{Ejercicio}

\title{Tarea 3. Álgebra y Matemática Discreta}
\author{Manu, Grupo 2.1}

\begin{document}
\maketitle

\begin{ejer}
Calcula los valores de $n$ enteros que cumplen simultáneamente las siguientes
tres relaciones:
\begin{align*}
15n-1 &\equiv 9 \pmod{125} \\
n     &\equiv 18 \pmod{64} \\
37n   &\equiv 58 \pmod{100}
\end{align*}
\end{ejer}

{\it Solución: }

% Escribe tu solución para el Ejercicio 1.
\begin{sageblock}
A = column_matrix(ZZ, [15, 125])
B = matrix(ZZ, [[1, 0], [0, 1]])
AB = block_matrix([ [A, B] ])
ABr = AB.echelon_form()

C = column_matrix(ZZ, [-25, 64])
D = matrix(ZZ, [[1, 0], [0, 1]])
CD = block_matrix([ [C, D] ])
CDr = CD.echelon_form()
CDr2 = CDr[:, 1:3]
CDr2t = CDr2.T
Pol.<t> = ZZ[]
E = column_matrix(Pol,[16 ,t])
D = CDr2t * E
\end{sageblock}
% Fin del ejercicio.
	
Simplificamos la ecuación y se queda
\begin{align*}
15n &= 10 \pmod{125}\\
\end{align*}

Para eliminar el 15 hay que comprobar que sea invertible en modulo 125, es decir que su maximo comun divisor sea 1.

\[\sage{AB} \to \sage{ABr} \]

Observamos que el maximo comun divisor es 5, por lo que dividimos ambas partes de la ecuacion entre 5.
\begin{align*}
\frac{15n}{5} = \frac{10 \pmod{125} }{5} \to 3n = 2 \pmod{25} \to n = 2 * 3^-1 \pmod{25} \to n = 2*17 \pmod{25} \to n = 9 \pmod{25}\\
\end{align*}

En la tercera ecuación


\begin{align*}
37n = 58 \pmod{100}\\
\end{align*}

Simplificamos y despejamos

\begin{align*}
n = 58 * 37^-1 \pmod{100} \to n = 58*73 \pmod{100} \to n = 46 \pmod{100}\\
\end{align*}

Ahora igualamos las ecuaciones 2 a 2.

Igualando la primera y la segunda ecuacion nos da

\begin{align*}
n = 9 \pmod{25} = 18 \pmod{64}\\
n = 9 + 25x1\\
n = 18 + 64x2\\
-25x1 + 64x2 = 34 - 18 = 16
\end{align*}

Reducimos por filas

\[\sage{CD} \to \sage{CDr} \]

Eso es igual a

\[\sage{CDr2t} \to \sage{E} \to \sage{D} = 64\]

Sustituimos

\begin{align*}
n = 34 + 25(64*t + 368) = 1600*t + 9234\\
n = 18 + 64(25*t + 144) = 1600*t + 9234\\
\end{align*}

El resultado final es

\begin{align*}
n = 9234 \pmod{1600} = 1234 \pmod{1600} 
\end{align*}

\end{document}
