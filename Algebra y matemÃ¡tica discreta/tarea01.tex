\documentclass{amsart}
\usepackage[margin = 2cm]{geometry}
\usepackage[utf8]{inputenc}
\usepackage[usefamily=sage]{pythontex} 
\usepackage{colortbl} 
\usepackage{xcolor}

\newtheorem{ejer}{Ejercicio}

\title{Tarea 1. Álgebra y Matemática Discreta}
\author{Manuel Bernal Hernández, Grupo 2.1}

\begin{document}
\maketitle

\begin{ejer}
Vamos a construir la matriz 

$$\left(\begin{array}{cccc|cc}
1 & \frac{1}{2} & 0 & 2  &         1 & 2 \\
0 &   0 & 1 & 3 &        -1 & 0 \\ \hline
0 &   1 & 0 & 0  &  \alpha_0 & \beta_0 \\
0 &   0 & 0 & 1  &  \alpha_1 & \beta_1 \\
\end{array}\right)$$

suponiendo que tenemos definida en {\tt sage} la matriz 
\begin{sageblock}
A = matrix(QQ,[[1,1/2,0,2,1,2],[0,0,1,3,-1,0]])
\end{sageblock}
$$ A = \sage{A}$$
y las variables $\alpha_0,\alpha_1,\beta_0,\beta_1$ son variables libres que
operan con los números racionales sobre los que está definida la matriz $A$.
En esta tarea vamos a usar un método ligeramente diferente al que hemos usado
en la práctica de clase. Seguiremos los siguientes pasos:

Empezaremos construyendo el anillo de polinomios con variables $\alpha_0,
\alpha_1, \beta_0$ y $\beta_1$ con coeficientes en ${\mathbb Q}$. 
Construiremos una matriz \verb|M| sobre este anillo de polinomios indicando 
Únicamente las dimensiones, por lo que la matriz estará llena de ceros.
Vamos a rellenar las tres partes de la matriz usando la asignación múltiple
que podemos ver en {\tt In [31]:} del archivo de comandos de {\tt sage} usado
para las prácticas de esta semana:

$$
\left(\begin{array}{cccc|cc} 
\cellcolor{red!40!white} 1 & \cellcolor{red!40!white}  \frac{1}{2} & \cellcolor{red!40!white} 0 & \cellcolor{red!40!white}  2 & \cellcolor{red!40!white} 1 & \cellcolor{red!40!white} 2 \\
\cellcolor{red!40!white} 0 & \cellcolor{red!40!white}            0 & \cellcolor{red!40!white} 1 & \cellcolor{red!40!white}  3 & \cellcolor{red!40!white} -1 & \cellcolor{red!40!white} 0 \\ \hline
\cellcolor{green!40!white} 0 & \cellcolor{green!40!lightgray} 1 & \cellcolor{green!40!white} 0 & \cellcolor{green!40!white} 0  & \cellcolor{yellow!60!white} \alpha_0 & \cellcolor{yellow!60!white} \beta_0 \\
\cellcolor{green!40!white} 0 & \cellcolor{green!40!white} 0 & \cellcolor{green!40!white} 0 & \cellcolor{green!40!lightgray} 1  & \cellcolor{yellow!60!white} \alpha_1 & \cellcolor{yellow!60!white} \beta_1 \\
\end{array}\right)
$$

La parte roja es la matriz $A$ que nos han dado, por lo que podemos asignar 
estos valores con \verb|M[:2,:] = A|. Para entender lo que significan estos Índices
fíjate en los ejemplos que aparecen en {\tt In [30]:} del archivo de comandos
de \verb|sage|.

Para poner los unos de la parte verde, como sólo son dos, asigna los valores 
directamente. Por Último para la parte amarilla, construye la matriz 
$$ B = \left(\begin{array}{cc} \alpha_0 & \beta_0 \\ \alpha_1 & \beta_1 
\end{array}\right)$$
y ponla en la posición correcta de la matriz \verb|M| del mismo modo que pusimos
la matriz $A$ con las coordenadas correctas. Si quieres completar la construcción
poniendo las líneas separadoras, utiliza el comando \verb|subdivide|.
\end{ejer}
{\it Solución: }

% Escribe tu solución para el Ejercicio 1.

\begin{sageblock}
P.<alpha0, alpha1, beta0, beta1> = QQ[]
A = matrix(QQ,[[1,1/2,0,2,1,2],[0,0,1,3,-1,0]])
B = matrix(P, [ [alpha0, beta0], [alpha1, beta1] ])
M = matrix(P, 4, 6)
M[:2,:] = A
M[2, 1] = 1
M[3, 3] = 1
M[2:4, 4:6] = B
M.subdivide(2, 4)
\end{sageblock}

La matriz M es: $$\sage{M}$$

La matriz M está construida por las matrices A y B, siendo A y B respectivamente $$\sage{A} \sage{B}$$

Las variables $\alpha_0,\alpha_1,\beta_0,\beta_1$ son variables libres


% Fin del ejercicio.

\end{document}


