\documentclass{amsart}
\usepackage[margin = 2cm]{geometry}
\usepackage[utf8]{inputenc}
\usepackage[usefamily=sage]{pythontex} 
\usepackage{colortbl} 
\usepackage{xcolor}

\newtheorem{ejer}{Ejercicio}

\title{Tarea 2. Álgebra y Matemática Discreta}
\author{Manuel Bernal Hernández. Grupo 2.1}

\begin{document}
\maketitle

\begin{ejer}
Encuentra todas las matrices $X$ tales que $AX = B$ siendo $A$ y $B$ las
siguientes matrices sobre los números reales dependientes de los parámetros
reales $a$ y $b$, 

\[
A = \left(\begin{array}{rrr}
2 & 0 & 2 \\
1 & 2 & -5 \\
-1 & -1 & 2 \\
0 & \frac{1}{2} & -\frac{3}{2}
\end{array}\right) \qquad
B = \left(\begin{array}{rrr}
b  & -2 & -4 \\
-a & 9 & 12 \\
a  & 2b & a+3b \\
0 & \frac{5}{2} & \frac{7}{2}
\end{array}\right)
\]
\end{ejer}
{\it Solución: }

% Escribe tu solución para el Ejercicio 1.
\begin{sageblock}
P.<a,b> = QQ[]
A = matrix(QQ, [[2, 0, 2], [1, 2, -5], [-1, -1, 2], [0, 1/2, -3/2]])
B = matrix(P, [[b, -2, -4], [-a, 9, 12], [a, 2*b, a + 3*b], [0, 5/2, 7/2]])
AB = block_matrix([ [A, B] ])
R = AB.echelon_form()
R1 = R(a = 1, b = -2)
R1.subdivide(2, 3)
R2 = R1[:2, :]
R2.subdivide([], 3)
P.<alpha1, alpha2, alpha3> = QQ[]
M = matrix(P, 3, 6)
M[:2, :] = R2
M[2, 2] = 1
M[2, 3] = alpha1
M[2, 4] = alpha2
M[2, 5] = alpha3
M.subdivide([], 3)
MR = M.echelon_form()
X = MR.subdivision(0, 1)
\end{sageblock}

Creamos la matriz ampliada y la reducimos por filas
$$ AB = \sage{AB} $$
$$ X = \sage{R} $$
Las dos últimas filas de la matriz nos dicen que para que se cumpla el sistema
$$ 2b + 4 = 0 $$
$$ \frac{a}{2} + \frac{b}{4} = \frac{a}{4} + \frac{b}{8} = 0 $$
$$ a + 3b + 5 = 0 $$
Ahora asignamos los valores a = 1 y b = 2 y sustituimos en la matriz
$$ \sage{R1} $$
% Fin del ejercicio.

\end{document}
