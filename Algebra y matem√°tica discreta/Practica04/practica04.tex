\documentclass{amsart}
\usepackage[margin = 2cm]{geometry}
\usepackage[utf8]{inputenc}
\usepackage[usefamily=sage]{pythontex}
\usepackage{colortbl} 
\usepackage{xcolor}
\usepackage[most]{tcolorbox}

\newtheorem{ejer}{Ejercicio}
\newtheorem{ejem}{Ejemplo}

\title{Práctica 4. Vectores Independientes y Generadores. Inversas laterales. Ecuaciones Implícitas y Paramétricas. }
\author{Escribe tu Nombre y Grupo}

\def\n{\mathbb{N}}
\def\r{\mathbb{R}}
\def\z{\mathbb{Z}}
\def\q{\mathbb{Q}}
\def\c{\mathbb{C}}

\begin{document}
\maketitle

\section{Vectores Independientes y Generadores}

Una combinación lineal de los vectores $v_1, v_2, \cdots, v_n$ es una 
expresión de la forma $x_1 v_1 + x_2 v_2 + \cdots + x_n v_n$ donde los
$x_i$ son escalares que llamaremos coeficientes de la combinación. Cuando
trabajemos con conjuntos de vectores, habitualmente los pondremos como 
columnas de una única matriz y en ese caso, las combinaciones lineales
se pueden escribir de forma compacta en lo que llamaremos representación
matricial de una combinación lineal:

\begin{sagecode}
latex.matrix_delimiters('[', ']')
A = matrix(QQ,[[1,0,1,2],[2,1,0,1],[3,1,4,2]])
Pol.<x1,x2,x3,x4> = QQ[]
X = matrix(Pol,[[x1],[x2],[x3],[x4]])
\end{sagecode}

$$ \underbrace{x_1 \sage{A[:,0]} + x_2 \sage{A[:,1]} + x_3 \sage{A[:,2]} + x_4 \sage{A[:,3]}}_{\text{Combinación Lineal en forma vectorial}} = \sage{A*X} = \underbrace{\sage{A} \sage{X}}_{\text{Combinación Lineal en forma matricial}} $$ 

Unos vectores se dicen {\bf linealmente independientes} si la única 
combinación  lineal que nos da el vector $0$ es la que tiene todos sus 
coeficientes  iguales a $0$. En forma matricial: Las columnas de $A$ son 
linealmente independientes si $AX = 0$ implica $X = 0$. Esto a veces también 
se denomina que $A$ es cancelable
\footnote{Esta terminología es para "cosas que se pueden multiplicar", no solo matrices. Hablando en general, todos los 
elementos invertibles son cancelables, pero puede haber elementos cancelables
no invertibles, por ejemplo, en la ecuación $3x=0$ en los números enteros,
nosotros sabemos que $x = 0$, aunque no podemos multiplicar por el inverso
de $3$ porque no existe en el conjunto de los números enteros. En el caso
concreto de matrices, cancelable por la izquierda va a ser equivalente a
tener inversa por la izquierda, pero eso lo veremos más adelante.}  
por la izquierda. 

La propiedad fundamental es que si $P$ es una matriz invertible, las 
columnas de $A$ son linealmente independientes si y solo si las columnas
de $PA$ son linealmente independientes. Si esto lo aplicamos a la matriz de
paso, $PA$ es la matriz reducida por filas. Entonces podemos decir que las
columnas de $A$ son linealmente independientes si y solo si las de su reducida
por filas lo son, y en las matrices reducidas por filas esa propiedad
es inmediata: En una matriz reducida por filas, sus columnas son
linealmente independientes si y solo si todas son columnas pivote (o lo que
es lo mismo: el {\bf rango}, que es el número de pivotes, coincide con el 
número de columnas). De hecho,
si de un conjunto de vectores queremos extraer un conjunto linealmente 
independiente con el mayor número de elementos posible, lo que tenemos que
hacer es reducir, ver qué columnas son pivote y esas columnas de la matriz
original serán linealmente independientes. Vamos a ver unos ejemplos:

\begin{ejem}
Para cada una de las siguientes matrices, determina si sus columnas son 
linealmente independientes. Si no lo son, encuentra un conjunto de columnas
linealmente independientes de dicha matriz que tenga el máximo número de 
elementos.

\begin{sagecode}
A = matrix(Zmod(5),[[2,2,3,1,2],[0,3,1,1,1],[2,3,0,4,1],[0,4,1,4,2]])
B = matrix(Zmod(5),[[4,1,0,2],[3,2,4,0],[4,0,1,0],[1,0,3,1],[2,1,3,0]])
C = matrix(Zmod(7),[[5,1,5,2,1],
[3,3,4,4,4],
[2,4,5,4,5],
[2,0,0,0,5],
[3,0,3,5,1]])
\end{sagecode}

\begin{enumerate}
\item $A = \sage{A}$ en el cuerpo ${\mathbb Z}_5$.
\item $B = \sage{B}$ en el cuerpo ${\mathbb Z}_5$.
\item $C = \sage{C}$ en el cuerpo ${\mathbb Z}_7$.
\end{enumerate}
\end{ejem}

{\it Solución:}
% Inicio del Ejemplo 1

\begin{enumerate}
\item 

\begin{sageblock}
A = matrix(Zmod(5),[[2,2,3,1,2],[0,3,1,1,1],[2,3,0,4,1],[0,4,1,4,2]])
AR = A.echelon_form()
\end{sageblock}

Reducimos por filas la matriz $A$.

$$ \sage{A} \to \sage{AR}$$

Como no todas las columnas son pivote los vectores no son linealmente
independientes. Las cuatro primera columnas sí lo son, y podemos tomar
$$ \sage{A[:,:4]}$$
como conjunto lineamente independiente con el máximo número de elementos.

\item 

\begin{sageblock}
B = matrix(Zmod(5),[[4,1,0,2],[3,2,4,0],[4,0,1,0],[1,0,3,1],[2,1,3,0]])
BR = B.echelon_form()
\end{sageblock}

Reducimos por filas la matriz $B$.

$$ \sage{B} \to \sage{BR}$$

Todas las columnas son pivote, por lo tanto estos vectores son linealmente
idependientes.

\item 
\begin{sageblock}
C = matrix(Zmod(7),[[5,1,5,2,1],[3,3,4,4,4],[2,4,5,4,5],[2,0,0,0,5],[3,0,3,5,1]])
CR = C.echelon_form()
\end{sageblock}

Reducimos por filas la matriz $C$.

$$ \sage{C} \to \sage{CR}$$

No todas las columnas son pivote, por lo tanto los vectores no son linealmente
idependientes. Si quitamos la cuarta columna que es la que nos ha salido "no
pivote" obtenemos un conjunto de vectores independientes con el número máximo
de elementos.

$$ \sage{C[:,[0,1,2,4]]}$$
\end{enumerate}

% Fin del Ejemplo 1

Dados $v_1, v_2, \cdots, v_n$ en el espacio $K^m$ para un cuerpo $K$, 
diremos que son {\bf generadores} si todo vector de $K^m$ se puede poner
como combinación lineal de dichos vectores. Si consideramos los vectores
$v_i$ como columnas de una matriz $B$, esta matriz tendrá tamaño $m$ filas
y $n$ columnas y la relación de ser generadores es equivalente a decir
que $BX=Y$ tiene solución para cualquier $Y$ de $K^m$. 

Si $B$ es una matriz $m\times n$ sobre $K$ y $E$ una operación elemental 
por filas, entonces las columnas de $B$ son generadoras (de $K^m$) si y 
solo si las de $EB$ son generadoras (de $K^m$). Este resultado que es muy 
fácil de demostrar nos permite reducir el problema de saber si las columnas
de $B$ son generadoras al de si su reducida por filas tiene sus columnas
generadoras. En el caso de una matriz reducida $R$, las columnas de $R$
son generadoras si y solo si $R$ no tiene ninguna fila de ceros, o lo que
es lo mismo, el número de pivotes (el rango) coincide con el número de filas.

\begin{ejem}
Para cada una de las siguientes matrices de tamaño $m \times n$ sobre el
cuerpo $K$, determina si sus columnas son generadoras de $K^m$.

\begin{sagecode}
A = matrix(Zmod(5),[[4,1,0,0,2,3,3],
[1,0,4,4,2,2,2],
[3,3,3,4,1,0,4],
[1,2,2,2,0,2,2],
[1,3,4,3,2,0,3]])
B = matrix(Zmod(5),[[0,2,3,3,1],
[0,0,4,3,0],
[4,0,4,1,0],
[2,3,1,0,4]])
C = matrix(Zmod(7),[[4,4,6,5],
[5,6,0,6],
[3,6,6,5],
[1,6,4,2],
[5,4,6,3]])
\end{sagecode}

\begin{enumerate}
\item $A = \sage{A}$ en el cuerpo ${\mathbb Z}_5$.
\item $B = \sage{B}$ en el cuerpo ${\mathbb Z}_5$.
\item $C = \sage{C}$ en el cuerpo ${\mathbb Z}_7$.
\end{enumerate}
\end{ejem}

{\it Solución:}
% Inicio del Ejemplo 2
\begin{enumerate}
\item Empecemos reduciendo por filas la matriz y viendo los pivotes que 
aparecen:

\begin{sageblock}
AR = A.echelon_form()
\end{sageblock}

$$ \sage{A} \to \sage{AR} $$

Estas columnas no son generadoras porque el rango (4) no coincide con el
número de filas (5).

\item Empecemos reduciendo por filas la matriz y viendo los pivotes que 
aparecen:

\begin{sageblock}
BR = B.echelon_form()
\end{sageblock}

$$ \sage{B} \to \sage{BR} $$

Estas columnas son generadoras porque el rango (4) coincide con el número
de filas (4).

\item Empecemos reduciendo por filas la matriz y viendo los pivotes que 
aparecen:

\begin{sageblock}
CR = C.echelon_form()
\end{sageblock}

$$ \sage{C} \to \sage{CR} $$

Estas columnas no son generadoras porque el rango no coincide con el número
de filas. En este caso no hubiese sido necesario ni siquiera hacer la 
reducción porque 4 columnas solo podrían tener como máximo 4 pivotes y 
nunca 5 (que el el número de filas).

\end{enumerate}
% Fin del Ejemplo 2

Los conjuntos que son al mismo tiempo linealmente independientes y generadores
se denominan {\bf bases}. De cualquier conjunto generador se puede extraer 
una base eligiendo un conjunto linealmente independiente con el máximo 
número de vectores (maximal). Dado cualquier conjunto de 
vectores, si le añadimos
un conjunto generador obtenemos evidentemente un conjunto generador, por lo 
tanto si a un conjunto linealmente independiente le añadimos una base (por 
ejemplo la canónica) obtendremos un conjunto generador. Si sacamos de este 
conjunto uno linealmente independiente mediante una reducción de matrices, 
obtendremos que 
las primeras columnas son pivotes (porque ya eran linealmente 
independientes) y luego nos aparecerán pivotes en la parte
que hemos añadido. Estas columnas pivote, si volvemos a la matriz original, 
formarán una base que amplía al conjunto independiente que nos han dado 
inicialmente. 

\begin{ejem}

\begin{sagecode}
A = matrix(Zmod(7),[[0,5,5],
[1,5,2],
[4,1,3],
[3,2,3],
[1,6,5]])
B = matrix(Zmod(7),[[4,6,6,5,0],
[6,6,1,6,5],
[1,5,5,2,4],
[0,2,0,0,4],
[1,4,2,6,2]])
\end{sagecode}

Comprueba que las columnas de la matriz $A$ son linealmente independientes y
que las de $B$ son generadoras de $K^5$ para el cuerpo $K = {\mathbb Z}_7$.
Amplía los vectores de $A$ para conseguir extender el conjunto a una base 
de $K^5$ usando los vectores que necesites de la base $B$.

$$ A = \sage{A} \qquad B = \sage{B} $$
\end{ejem}

{\it Solución:}
% Inicio del Ejemplo 3

\begin{sageblock}
AR = A.echelon_form()
BR = B.echelon_form()
AB = block_matrix([[A,B]])
ABR = AB.echelon_form()
\end{sageblock}

Para responder a las preguntas del ejercicio vamos a reducir las matrices
$A$ y $B$ por filas

$$ \sage{A} \to \sage{AR}$$

Los vectores de $A$ son independientes porque su reducida tiene pivotes 
en todas las columnas.

$$ \sage{B} \to \sage{BR}$$

Los vectores de $B$ son independientes y generadores (y por tanto una base) 
porque al reducir, el número de pivotes coincide con el número de columnas y 
de filas. 
Para la última parte construimos el conjunto de vectores formados por las
columnas de la matriz $A$ y los de la matriz $B$. Este conjunto será 
claramente generador porque los de $B$ ya son generadores. Si reducimos por
filas obtendremos lo siguiente:

$$ \sage{AB} \to \sage{ABR} $$

Esto nos muestra que si añadimos el primer y el tercer vector de $B$ (que nos
dan los pivotes que nos faltan) obtendríamos un conjunto linealmente 
idependiente y generador que amplía los vectores de $A$. Los vectores que
nos piden son 
$$ \sage{AB[:,[0,1,2,3,5]]}$$
Podemos comprobar que es realmente es base reduciendo esta matriz
$$ 
\sage{AB[:,[0,1,2,3,5]]} \to 
\sage{AB[:,[0,1,2,3,5]].echelon_form()}. 
$$

% Fin del Ejemplo 3

\begin{ejer}
Para cada una de las matrices siguientes de tamaño $m \times n$ sobre el 
cuerpo $K$ determina si las columnas de la matriz son linealmente 
independientes, generadores de $K^m$ y/o bases de $K^m$. En el caso de 
los conjuntos generadores que no sean base, extrae una base de los mismos.
El el caso de los linealmente independientes que no son bases, amplíalos a
bases utilizando los vectores que sean necesarios de la base canónica.

\begin{sagecode}
A = matrix(Zmod(5),[[4,2,4,0,1],
[1,1,0,4,4],
[3,0,1,2,3],
[1,2,0,1,3],
[2,4,1,2,2]])
B = matrix(Zmod(5),[[3,0,2,1,3],
[2,0,3,2,1],
[1,0,1,4,2],
[4,3,2,4,3],
[0,2,4,2,0]])
C = matrix(Zmod(7),[[6,2,0,5,5],
[2,2,4,2,0],
[1,3,2,5,1]])
D = matrix(Zmod(7),[[1,1,4,0,1],
[5,4,2,3,1],
[5,2,1,2,0]])
E = matrix(QQ,[[-2,0,1],
[-1,1/2,0],
[0,1,-1],
[-2,2,2],
[-1,0,1]])
F = matrix(QQ,[[-2,0,-1,2],
[-2,-2,0,-1],
[0,-2,1,0],
[-3,-1,-1,-2],
[-1,-1,0,0],
[0,-1,1/2,-1]])
\end{sagecode}

\begin{enumerate}
\item $ A = \sage{A}$ sobre $K = {\mathbb Z}_5$.
\item $ B = \sage{B}$ sobre $K = {\mathbb Z}_5$.
\item $ C = \sage{C}$ sobre $K = {\mathbb Z}_7$.
\item $ D = \sage{D}$ sobre $K = {\mathbb Z}_7$.
\item $ E = \sage{E}$ sobre $K = {\mathbb Q}$.
\item $ F = \sage{F}$ sobre $K = {\mathbb Q}$.
\end{enumerate}
\end{ejer}
{\it Solución:}

% Escribe tu solución para el Ejercicio 1

% Fin del Ejercicio 1


\section{Inversas Laterales}

Dos matrices cuadradas $A$ y $B$ se dicen inversas una de la otra si 
$AB = I = BA$. Sabemos que en el conjunto de las matrices el producto
no es conmutativo, sin embargo en el caso de las matrices cuadradas, si
$AB = I$ podemos estar seguros de que $BA = I$. Otra propiedad importante
que tienen las matrices cuadradas es que si una matriz tiene inversa, 
esa inversa es única, no pueden haber dos inversas diferentes para la 
misma matriz cuadrada.

En el caso de las matrices no cuadradas tenemos la posibilidad de que 
dos matrices $A$ y $B$ cumplan que $AB = I$ pero $BA$ no sea la identidad.
En ese caso diremos que $A$ es inversa por la izquierda de $B$ o que $B$ 
es inversa por la derecha de $A$. En matrices no cuadradas las inversas
laterales no son únicas y si son inversas por un lado, no lo pueden ser 
por el otro. 

Si nos dan una matriz $A$ para la que queremos encontrar, por ejemplo, 
sus inversas por la derecha, podríamos plantear la ecuación matricial 
$AX = I$ y despejar $X$ utilizando las técnicas de la Práctica 2, pero
para este caso particular existe otra técnica que puede resultar más
efectiva y que también nos ayudará a entender otros problemas posteriores.

Antes de iniciar el ejemplo, vamos a darnos cuenta de que encontrar las
inversas laterales por la izquierda o la derecha son problemas simétricos
y si sabemos resolver uno de ellos, el otro es inmediato ya que si $AB = I$
entonces $B^t A^t = (AB)^t = I^t = I$. Por lo tanto, tomando traspuestas
podemos reducir un problema a otro de forma inmediata.

\begin{ejem}
Determina si las siguientes matrices tienen inversas laterales y en 
el caso en que las tengan, calcúlalas:

\begin{sagecode}
M = matrix(QQ,[[2,2,0,0],
[2,2,0,1],
[4,-2,0,3],
[1,1,0,-1],
[2,3,0,0],
[0,3,1,-2],
[2,1,0,1]])
N = matrix(QQ,[[-2,0,-1,0,-1,-2,0],
[1,0,0,0,0,0,0],
[0,-1,0,-1,0,0,0],
[2,0,0,0,0,1,0]])
\end{sagecode}

\begin{enumerate}
\item $M = \sage{M}$ sobre el cuerpo ${\mathbb Q}$.
\item $N = \sage{N}$ sobre el cuerpo ${\mathbb Q}$.
\end{enumerate}
\end{ejem}

{\it Solución:}
% Inicio del Ejemplo 4

\begin{enumerate}
\item Ampliamos la matriz con la matriz identidad a la derecha y 
reducimos por filas:

\begin{sageblock}
M1 = block_matrix([[M,1]])
M1R = M1.echelon_form()
\end{sageblock}

$$ \sage{M1} \to \sage{M1R}$$

La condición necesaria y suficiente para tener una inversa por la izquierda
es que al reducir, todas las columnas sean pivote (o lo que es lo mismo,
que los vectores sean linealmente independientes). Entonces tendremos
una matriz reducida con la siguiente estructura
$$ \left[\begin{array}{c|c} I & A \\ \hline 0 & H \end{array}\right]$$
En nuestro ejemplo tendríamos

\begin{sageblock}
M1R = copy(M1R)
M1R.subdivide(4,4)
A = M1R.subdivision(0,1)
H = M1R.subdivision(1,1)
\end{sageblock}

$$\sage{M1R}$$
$$ A = \sage{A} $$
$$ H = \sage{H} $$
El teorema fundamental de la reducción por filas nos dice que 
$$ \sage{A} \sage{M} = \sage{A*M} $$
$$ \sage{H} \sage{M} = \sage{H*M} $$

La matriz $A$ vemos que es una posible inversa por la izquierda, pero 
hay muchas, cualquiera que podamos poner de la forma $A+CH$ con $C$
cualquier matriz de parámetros libres cumplirá que
$$ (A+CH)M = AM + CHM = I + C0 = I.$$
Como $A$ tiene tamaño $4 \times 7$ y $H$ tiene tamaño $3 \times 7$
el tamaño de la matriz $C$ debe ser $4 \times 3$ para que la operación
sea posible.

\begin{sageblock}
Pol.<a1,a2,a3,b1,b2,b3,c1,c2,c3,d1,d2,d3> = QQ[]
C = matrix(Pol,[[a1,a2,a3],[b1,b2,b3],[c1,c2,c3],[d1,d2,d3]])
\end{sageblock}

La solución general es pues
$$ \sage{A} + \sage{C} \sage{H} = $$
$$ \sage{A+C*H}$$
donde los $a_i,b_i,c_i,d_i$ son $12$ parámetros libres en ${\mathbb Q}$.

\item En este segundo ejemplo tenemos una matriz con más columnas que filas.
Este tipo de matrices puede tener inversas laterales por la derecha y para
encontrarlas tenemos que hacer el mismo proceso que antes pero por columnas,
o lo que es lo mismo, resolviendo el problema usando traspuestas. La regla
es sencilla, al ampliar con la matriz identidad, esta debe estar siempre por 
el lado más grande de la matriz así que si la matriz es "horizontal" tenemos
que tomar traspuestas para amplicar con la identidad a la derecha.

\begin{sageblock}
Nt1 = block_matrix([[N.T,1]])
Nt1R = Nt1.echelon_form()
\end{sageblock}
\end{enumerate}

$$ \sage{Nt1} \to \sage{Nt1R}$$

Como todas las columnas son pivote, la matriz $N^t$ tendrá inversa por
la izquierda (y por lo tanto $N$ tendrá inversa por la derecha). Esta 
condición es equivalente a que las columnas de $N$ sean generadoras.

\begin{sageblock}
Nt1R = copy(Nt1R)
Nt1R.subdivide(4,4)
A = Nt1R.subdivision(0,1)
H = Nt1R.subdivision(1,1)
\end{sageblock}

La matriz $\sage{A}$ es una inversa por la izquierda de $N^t$ y la fórmula
general para todas las matrices inversas por la izquierda de $N^t$ es
$$ \sage{A} + C \sage{H} $$
donde $C$ es una matriz de parámetros libres de tamaño $4 \times 3$.

\begin{sageblock}
Pol.<a1,a2,a3,b1,b2,b3,c1,c2,c3,d1,d2,d3> = QQ[]
C = matrix(Pol,[[a1,a2,a3],[b1,b2,b3],[c1,c2,c3],[d1,d2,d3]])
Sol = (A+C*H).T
\end{sageblock}

Las inversas por la derecha de $N$ serán por lo tanto
$$ \left( \sage{A} + \sage{C}\sage{H}\right)^t = $$
$$ \sage{Sol} $$
siendo $a_i,b_i,c_i$ y $d_i$ parámetros libres en ${\mathbb Q}$.
Podemos comprobar el resultado calculando el producto de estas soluciones
por la matriz $N$,
$$ \sage{N}\sage{Sol} = \sage{N*Sol}$$

% Fin del Ejemplo 4

\begin{ejer}
Para cada una de las siguientes matrices, determina si tienen inversas por
la izquierda, por la derecha o por ambos lados. En caso de tenerlas, 
calcúlalas.

\begin{sagecode}
M1 = matrix(Zmod(5),[[0,1,4,0,0,3],
[3,3,3,4,4,3],
[3,4,3,3,4,3],
[1,4,3,3,2,0]])
M2 = matrix(Zmod(5),[[0,0,2,1,1,1],
[2,3,3,1,4,2],
[1,4,4,1,4,4],
[3,2,2,2,3,1]])
M3 = matrix(Zmod(5),[[2,3,4],
[2,0,3],
[1,4,1],
[1,3,4],
[0,4,1],
[4,2,1]])
M4 = matrix(Zmod(5),[[4,1,3],
[3,1,2],
[0,1,4],
[3,0,3],
[3,0,3],
[3,1,2]])
M5 = matrix(Zmod(5),[[1,1,3,0],
[4,4,2,3],
[2,4,1,2],
[1,3,3,0]])
M6 = matrix(Zmod(5),[[3,2,3,2],
[4,1,0,1],
[0,2,0,4],
[4,0,0,3]])
\end{sagecode}

\begin{enumerate}
\item $M_1 = \sage{M1}$ sobre el cuerpo ${\mathbb Z}_5$.
\item $M_2 = \sage{M2}$ sobre el cuerpo ${\mathbb Z}_5$.
\item $M_3 = \sage{M3}$ sobre el cuerpo ${\mathbb Z}_5$.
\item $M_4 = \sage{M4}$ sobre el cuerpo ${\mathbb Z}_5$.
\item $M_5 = \sage{M5}$ sobre el cuerpo ${\mathbb Z}_5$.
\item $M_6 = \sage{M6}$ sobre el cuerpo ${\mathbb Z}_5$.
\end{enumerate}

\end{ejer}

{\it Solución:}

% Escribe tu solución para el Ejercicio 2

% Fin del Ejercicio 2

\section{Ecuaciones Implícitas y Paramétricas}

Sea $K$ un cuerpo y $m$ un número natural. Los elementos de $K^m$ los 
llamaremos vectores y un espacio vectorial $V$ sobre el cuerpo $K$ es un 
subconjunto no vacío de $K^m$ cumpliendo las siguientes condiciones:
\begin{itemize}
\item Si $u,v \in V$ entonces $u+v \in V$.
\item Si $u \in V$ y $\alpha \in K$, entonces $\alpha v \in V$.
\end{itemize}
Existen dos formas fundamentales para definir un espacio vectorial a 
partir de una matriz:

\subsection{Forma Paramétrica} Dada una matriz $B$, llamaremos\footnote{La
notación $C(B)$ proviene del nombre en inglés {\it column space} en español
se denomina espacio generado por las columnas de $B$.} $C(B)$
al conjunto de todos los vectores que se pueden poner como combinación
lineal de las columnas de $B$. Cuando un espacio vectorial $V$ nos lo
dan como $V = C(B)$ diremos que $V$ nos lo están dando en forma paramétrica.

\begin{sagecode}
Pol.<alpha1,alpha2,alpha3>  = Zmod(2)[]
B = matrix(Zmod(2),[[1,0,1],
[1,1,1],
[1,0,1],
[0,1,0],
[0,0,1]])
V = []
for a1 in Zmod(2):
  for a2 in Zmod(2):
    for a3 in Zmod(2):
      v = a1*B[:,0]+a2*B[:,1]+a3*B[:,2]
      if v not in V:
        V.append(v)
\end{sagecode}

Si $B = \sage{B}$ sobre ${\mathbb Z}_2$ enconces
$C(B)$ es el conjunto de los vectores que se pueden escribir como
$$ \alpha_1 \sage{B[:,0]} + \alpha_2 \sage{B[:,1]} + \alpha_3\sage{B[:,2]}$$
para $\alpha_1,\alpha_2,\alpha_3$ todos los valores posibles en 
${\mathbb Z}_2$. Si tomamos todos los valores, obtenemos que 

$$V = C(B) = \left\{ \sage{V[0]},\sage{V[1]},\sage{V[2]},\sage{V[3]},
\sage{V[4]},\sage{V[5]},\sage{V[6]},\sage{V[7]}\right\}$$  

\subsection{Forma Implícita} Dada una matriz $H$, llamaremos\footnote{La
notación $N(H)$ proviene del nombre en inglés {\it null space} en español
se denomina espacio anulador por la derecha de $H$.} $N(H)$ al conjunto
de vectores $X$ tales que $HX = 0$. Cuando un espacio vectorial $V$ nos lo
dan como $V = N(H)$ diremos que $V$ nos lo están dando en forma implícita.

\begin{sagecode}
Pol.<x1,x2,x3,x4,x5> = Zmod(2)[]
H = matrix(Zmod(2),[[1,0,1,0,0],
[0,1,1,1,0]])
X = matrix(Pol,[[x1],[x2],[x3],[x4],[x5]])
V = [column_matrix(v) for v in VectorSpace(Zmod(2),5) if H*v == 0]
\end{sagecode}

Por ejemplo, si $H = \sage{H}$ sobre ${\mathbb Z}_2$ entonces
$N(H)$ es el conjunto de los vectores $X = \sage{X}$ que se pueden escribir
como $HX = 0$, es decir, 
$$ \sage{H} \sage{X} = \sage{H*X} = \sage{matrix(Zmod(2),2,1)}$$
que separado en forma de dos ecuaciones nos dice que
\begin{align*} 
\sage{(H*X)[0,0]} &= 0 \\
\sage{(H*X)[1,0]} &= 0 \\
\end{align*}
Encontrando todos los valores posibles, obtenemos

$$V = N(H) = \left\{ \sage{V[0]},\sage{V[1]},\sage{V[2]},\sage{V[3]},
\sage{V[4]},\sage{V[5]},\sage{V[6]},\sage{V[7]}\right\}$$  

\vspace{1cm}

Tal y como hemos visto en este ejemplo, un mismo espacio vectorial se puede
poner tanto de forma implícita como de forma paramétrica. Pasar de una 
representación a otra será una parte fundamental de muchos de los problemas
porque la utilización de la representación adecuada permitirá resolverlos
de forma sencilla mientras que con la representación incorrecta el problema
puede ser muy complicado. Vamos a aprender en esta sección a pasar de una
representación a la otra con unos ejemplos:

\begin{sagecode}
B = matrix(Zmod(5),[[2,4,2],
[2,0,3],
[4,2,3],
[0,4,4],
[4,0,1]])
\end{sagecode}

\begin{ejem}
Dada la matriz $B = \sage{B}$ sobre el cuerpo ${\mathbb Z}_5$ describe el 
espacio vectorial $V = C(B)$ en forma implícita.
\end{ejem}

{\it Solución:}
% Inicio del Ejemplo 5

\begin{sageblock}
B1 = block_matrix([[B,1]])
B1R = copy(B1.echelon_form())
B1R.subdivide(2,3)
H = B1R.subdivision(1,1)
\end{sageblock}

Ampliamos la matriz con la identidad a la derecha reducimos por filas
$$ \sage{B1} \to \sage{B1R} $$
Si denominamos $H$ a la parte que queda a la derecha de los ceros al hacer la
reducción tenemos que $H = \sage{H}$. El teorema fundamental de la reducción
por filas nos dice que $HB = 0$ y por lo tanto 
$$ 
\sage{H} \sage{B[:,0]} = \sage{H*B[:,0]},
\sage{H} \sage{B[:,1]} = \sage{H*B[:,1]},
\sage{H} \sage{B[:,2]} = \sage{H*B[:,2]}$$
Si eso es cierto para las columnas de $B$, lo es para cualquier combinación
lineal de las columnas de $B$ (para todos los vectores de $C(B)$) y la matriz
$H$ es precisamente la matriz que nos da $V$ de forma implícita.

$$ N\left(\sage{H}\right) = V = C\left(\sage{B}\right).$$

% Fin del Ejemplo 5

\vspace{1cm}

El problema inverso se puede resolver de forma dual usando traspuestas. La 
razón es una propiedad muy útil que es 
$$ N(H) = C(B) ~\text{si y solo si}~ C(H^t) = N(B^t)$$
Esta propiedad la veremos de nuevo cuando hablemos de espacio ortogonal, pero
de momento nos servirá para pasar de implícitas a paramétricas usando la misma
técnica que para pasar de paramétricas a implícitas.

\begin{sagecode}
H = matrix(Zmod(7),[[2,2,6,6,6],
[0,2,2,3,5],
[5,0,6,4,4]])
Pol.<x1,x2,x3,x4,x5> = Zmod(7)[]
X = matrix(Pol,[[x1],[x2],[x3],[x4],[x5]])
\end{sagecode}

\begin{ejem}
Sea $V$ el espacio vectorial sobre el cuerpo ${\mathbb Z}_7$ formado por los
$X$ tales que 
\begin{align*}
\sage{(H*X)[0,0]} &= 0 \\
\sage{(H*X)[1,0]} &= 0 \\
\sage{(H*X)[2,0]} &= 0 \\
\end{align*}
Escribe $V$ en forma paramétrica.
\end{ejem}

{\it Solución:}
% Inicio del Ejemplo 6

\begin{sageblock}
Ht1 = block_matrix([[H.T,1]])
Ht1R = copy(Ht1.echelon_form())
Ht1R.subdivide(3,3)
B = Ht1R.subdivision(1,1).T
\end{sageblock}

Lo primero que tenemos que hacer es poner el espacio como $HX = 0$, lo cual
podemos hacer tomando $H = \sage{H}$.
Trasponemos $H$ y ampliamos la matriz con la identidad para reducir por filas
$$ \sage{Ht1} \to \sage{Ht1R} $$
Si denominamos $B$ a la traspuesta de la matriz que queda en la parte derecha 
de los ceros al hacer la reducción tenemos que
$$ B = \sage{B} $$
Entonces tenemos que
$$ N\left(\sage{H}\right) = V = C\left(\sage{B}\right).$$
% Fin del Ejemplo 6

\begin{sagecode}
B = matrix(Zmod(3),[[1,0,1,2],
[0,1,0,2],
[1,1,2,2],
[0,1,0,1],
[0,0,2,0],
[2,0,2,1]])
H = matrix(Zmod(7),[[5,6,2,3,1],
[2,0,0,3,1],
[4,1,5,0,0]])
\end{sagecode}

\begin{ejer}
\begin{enumerate}
\item Sea $B = \sage{B}$ sobre el cuerpo ${\mathbb Z}_3$ y $V = C(B)$. Escribe $V$
en forma implícita.
\item Sea $H = \sage{H}$ sobre el cuerpo ${\mathbb Z}_7$ y $W = N(H)$. Escribe $W$
en forma paramétrica.
\end{enumerate}
\end{ejer}

{\it Solución:}

% Escribe tu solución para el Ejercicio 3

% Fin del Ejercicio 3


\end{document}



