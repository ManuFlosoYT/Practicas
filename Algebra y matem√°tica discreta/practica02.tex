\documentclass{amsart}
\usepackage[margin = 2cm]{geometry}
\usepackage[utf8]{inputenc}
\usepackage[usefamily=sage]{pythontex}
\usepackage{colortbl} 
\usepackage{xcolor}

\newtheorem{ejer}{Ejercicio}
\newtheorem{ejem}{Ejemplo}

\title{Práctica 2. Ecuaciones Lineales y Matriciales}
\author{Escribe tu Nombre y Grupo}

\def\n{\mathbb{N}}
\def\r{\mathbb{R}}
\def\z{\mathbb{Z}}
\def\q{\mathbb{Q}}
\def\c{\mathbb{C}}

\begin{document}
\maketitle

La forma más simple de lo que entendemos como una ecuación es una expresión 
del tipo 
$$ 3x = 12 $$
donde $x$ es lo que llamamos incógnita. Lo que multiplica a la $x$ se denomina
coeficiente y la parte derecha se llamará término independiente. Resolver una
ecuación consiste en hacer transformaciones hasta conseguir que el coeficiente
se convierta en $1$ y obtengamos el valor de $x$. En el caso de la ecuación 
anterior, lo que hacemos es multiplicar por $\frac{1}{3}$ los dos miembros de
la ecuación. Con eso tenemos 
$$ x = 1 \cdot x = \frac{1}{3} \cdot 3 \cdot x = \frac{1}{3} \cdot 12 = 4. $$
Un sistema de ecuaciones lineales es una expresión de la forma 
\begin{align*}
x_{1} - x_{2} - 5 x_{3} - 5 x_{4} &= 6 \\
-x_{1} - x_{2} + 2 x_{3} + x_{4} &= 1 \\
-x_{2} - 2 x_{3} - 3 x_{4} &= 5 \\
x_{4} &= -2 \\
\end{align*}
Esta expresión nos puede parecer muy distinta a la ecuación anterior, pero
podemos escribir en forma matricial como 
$$ AX = B $$
donde ahora $A,X$ y $B$ son matrices. Concretamente 
$$ A = \left(\begin{array}{rrrr}
1 & -1 & -5 & -5 \\
-1 & -1 & 2 & 1 \\
0 & -1 & -2 & -3 \\
0 & 0 & 0 & 1
\end{array}\right) \quad
X = \left(\begin{array}{r} x_1 \\ x_2 \\ x_3 \\ x_4 \end{array}\right) \quad
B = \left(\begin{array}{r} 6 \\ 1 \\ 5 \\ -2 \end{array}\right)$$
La forma matricial será la preferida por nosotros a lo largo de este curso. 
La matriz $A$ se llama {\bf matriz de los coeficientes}, $X$ la {\bf matriz de
incógnitas} y $B$ la {\bf matriz de términos independientes}. Para representar
esta ecuación matricial se utiliza la {\bf matriz ampliada del sistema} que 
denotaremos $[A|B]$ y es la formada por dos bloques, uno con la matriz $A$ y 
otro con la matriz $B$. En este caso 

$$[A|B] = \left(\begin{array}{rrrr|r}
1 & -1 & -5 & -5 & 6 \\
-1 & -1 & 2 & 1 & 1 \\
0 & -1 & -2 & -3 & 5 \\
0 & 0 & 0 & 1 & -2
\end{array}\right)$$

El objetivo de la matriz ampliada del sistema es simplificar las operaciones 
que tenemos que hacer para resolver la ecuación. La idea es la misma que en la
ecuación $3x = 12$, vamos a hacer operaciones en ambos miembros de la ecuación
hasta que consigamos que lo que multiplica a $X$ sea $1$ (o en este caso la 
matriz identidad). Las operaciones que podemos hacer deben ser operaciones que
nos mantengan el mismo conjunto de soluciones. Dicho de otra forma, si hacemos
alguna operación a la matriz $[A|B]$ y obtenemos la matriz $[A'|B']$, las 
matrices $X$ tales que $AX = B$ deben ser las mismas que las que hacen $A'X = B'$.
Esto sucede por ejemplo cuando hacemos una operación elemental por filas $E$
porque tal y como hemos visto en teoría eso es lo mismo que multiplicar por
la matriz elemental $E$ y todas las matrices elementales son invertibles. 
Entonces $AX = B$ si y solo si $EAX = EB$. El proceso que simplifica la matriz
$A$ utilizando operaciones elementales es la {\bf reducción por filas}. El caso
más sencillo es que ese proceso termine dejando a la izquierda la matriz identidad
$$[A|B] \mapsto [I|B']$$
porque entonces tenemos que $X = IX = B'$ y ha hemos resuelto la ecuación. 
En este caso diremos que el sistema es {\bf compatible determinado}.
Veámoslo en el primer ejemplo:

\begin{ejem} Estudia y resuelve si es posible el siguiente sistema de ecuaciones 
sobre $\r $ :
\begin{align*}
x_{1} - x_{2} - 5 x_{3} - 5 x_{4} &= 6 \\
-x_{1} - x_{2} + 2 x_{3} + x_{4} &= 1 \\
-x_{2} - 2 x_{3} - 3 x_{4} &= 5 \\
x_{4} &= -2 \\
\end{align*}
\end{ejem}

{\it Solución:}
% Inicio del ejemplo 1

Aunque el enunciado habla de $\r$, vamos a usar los números racionales $\q$ 
para conseguir resultados exactos y evitar decimales. 

\begin{sageblock}
A = matrix(QQ,[[1,-1,-5,-5],[-1,-1,2,1],[0,-1,-2,-3],[0,0,0,1]])
B = matrix(QQ,[[6],[1],[5],[-2]])
AB = block_matrix([[A,B]])
R = AB.echelon_form()
\end{sageblock}

La matriz ampliada del sistema es:

\[[A|B] = \sage{AB} \]

Reducimos por filas obteniendo la matriz

\[ \sage{R} \]

Como la matriz que nos queda a la izquierda es la matriz identidad, la que nos
queda a la derecha es la solución $X$ que estamos buscando. Es decir, 

$$ X = \sage{R.subdivision(0,1)} $$

o lo que es lo mismo, 

\[ x_1 = 0, \ \ x_2 = -1,  \ \ x_3 = 1,  \ \ x_4 = -2.\]

% fin del ejemplo 1

Resuelve el siguiente ejercicio siguiendo el mismo método del ejemplo anterior.

\begin{ejer}
Estudia y resuelve si es posible el siguiente sistema de ecuaciones sobre el 
cuerpo ${\mathbb Z}_5$.

\begin{align*}
3 x_{2} + x_{3} &= 0\\
x_{1} - x_{2} + 3 x_{3} + 3 x_{4} &= 3\\
3 x_{2} - x_{3} + 3 x_{4} &= 4 \\
3 x_{1} + x_{2} + x_{3} + x_{4} &= 3\\
\end{align*}
\end{ejer}
{\it Solución:}

% Escribe tu solución al Ejercicio 1
\begin{sageblock}
A = matrix(Zmod(5),[[0, 3, 1, 0],[1, -1, 3, 3],[0, 3, -1, 3],[3, 1, 1, 1]])
B = matrix(Zmod(5),[[0],[3],[4],[3]])
AB = block_matrix([[A,B]])
R = AB.echelon_form()
X = R.subdivision(0, 1)
\end{sageblock}
% Fin del Ejercicio 1

\vspace{1cm}

Vamos a ver otros casos que nos pueden aparecer cuando hacemos la reducción
de la matriz ampliada el sistema. Concretamente vamos a ver qué pasa cuando 
aparecen filas de ceros.  

\begin{ejem} Estudia y resuelve si es posible el siguiente sistema de 
ecuaciones sobre $\r $
\begin{align*}
2 x_{1} - x_{2} &= 8 \\
x_{1} - x_{3} &= -3 \\
x_{2} &= -5 \\
x_{1} &= 2 \\
x_{3} &= 8 \\
\end{align*}
\end{ejem}

{\it Solución:}

% Inicio del Ejemplo 2

\begin{sageblock}
A = matrix(QQ,[[2,-1,0],[1,0,-1],[0,1,0],[1,0,0],[0,0,1]])
B = matrix(QQ,[[8],[-3],[-5],[2],[8]])
AB = block_matrix(1,2,[A,B])
R = copy(AB.echelon_form())
R.subdivide(3,3)
\end{sageblock}

Construimos la matriz ampliada del sistema y la reducimos por filas

\[[A|B] = \sage{AB} \to \sage{R} \]

Como la columna de términos independientes es una columna pivote, el 
sistema es {\bf incompatible}, pero vamos a interpretar esta matriz reducida
desde el punto de vista de las operaciones con matrices.

\begin{sageblock}
A1 = R.subdivision(0,0)
A2 = R.subdivision(1,0)
B1 = R.subdivision(0,1)
B2 = R.subdivision(1,1)
\end{sageblock}

La matriz reducida indica que buscamos las matrices $X$ tales que
\[ \sage{R[:,:3]} X = \sage{R[:,3]} \]
esta ecuación se puede dividir en dos ecuaciones
\[ \sage{A1} X = \sage{B1} \qquad \sage{A2} X = \sage{B2}\]
La primera ecuación no da ningún problema, pero la segunda sí porque 
$0$ multiplicado por cualquier cosa es $0$, por lo tanto la segunda 
igualdad es imposible y por eso el sistema no puede tener solución.

% Fin del Ejemplo 2

El sistema será incompatible en cuanto tengamos que algo multiplicado
por $0$ tiene un valor distinto de cero, independientemente de lo que
nos digan el resto de las ecuaciones. Esto se refleja en que siempre
que la parte derecha de la matriz tiene algún pivote, entonces el sistema
es incompatible como se puede ver en este ejemplo:

\begin{ejem} Estudia y resuelve si es posible el siguiente sistema de ecuaciones sobre $\z _5$:
\[ 0 = 0 \]
\[ x_{1} + 3 x_{2} + 2 x_{3} = 0 \]
\[ -x_{1} + x_{2} + x_{3} = 1 \]
\[ -x_{1} + 3 x_{2} = 3 \]
\[ -x_{1} + 3 x_{2} = 0 \]
\end{ejem}

{\it Soluci\'on: }
% Inicio del Ejemplo 3

\begin{sageblock}
A = matrix(Zmod(5),[[0,0,0],[1,3,2],[4,1,1],[4,3,0],[4,3,0]])
B = matrix(Zmod(5),[[0],[0],[1],[3],[0]])
AB = block_matrix(1,2,[A,B])
R = AB.echelon_form()
\end{sageblock}

Determinamos la matriz ampliada del sistema 

\[[A|B] = \sage{AB} \]

Reducimos por filas obteniendo la matriz

\[ \sage{R} \]

Como la columna de t\'erminos independientes es una columna pivote, el sistema es incompatible.

% Fin del Ejemplo 3

También podemos tener filas de ceros pero con la parte derecha igualada
a $0$ como en el siguiente ejemplo:

\begin{ejem} Estudia y resuelve si es posible el siguiente sistema de ecuaciones sobre $\r $ :
\begin{align*}
-x_{2} + 3 x_{3} &= 3 \\
x_{1} - 2 x_{2} + 5 x_{3} &= 3 \\
x_{1} - 3 x_{2} + 9 x_{3} &= 7 \\
x_{1} - 2 x_{2} + 3 x_{3} &= 1 \\
2 x_{1} - 3 x_{2} + 9 x_{3} &= 5 \\
\end{align*}
\end{ejem}

{\it Soluci\'on:}
% Inicio del Ejemplo 4

\begin{sageblock}
A = matrix(QQ,[[0,-1,3],[1,-2,5],[1,-3,9],[1,-2,3],[2,-3,9]])
B = matrix(QQ,[[3],[3],[7],[1],[5]])
AB = block_matrix(1,2,[A,B])
R = AB.echelon_form()
\end{sageblock}

Determinamos la matriz ampliada del sistema 

\[[A|B] = \sage{AB} \]

Reducimos por filas obteniendo la matriz

\[ \sage{R} \]

Como todas las columnas de la parte de los coeficientes resultan ser columnas pivote, 
el sistema es compatible determinado. Las dos filas de ceros que tenemos 
debajo lo único que nos dicen es que 

\[ \sage{R[3:,:3]} X = \sage{R[3:,3]}\]
lo cual no aporta nada, por eso esta matriz reducida es totalmente equivalente a 

\[ \sage{R[:3,:]} \] y por eso el sistema es compatible determinado. 
La solución es

\[ X = \sage{R[:3,3]}. \]

o lo que es lo mismo

\[ x_1 = -2, \ \ x_2 = 0,  \ \ x_3 = 1.\]

% Fin del Ejemplo 4

Realiza los siguientes ejercicios del mismo modo que los ejemplos
mostrados.

\begin{ejer}
Estudia y resuelve si es posible el siguiente sistema de ecuaciones 
sobre el cuerpo ${\mathbb Z}_5$.

\begin{align*}
3 x_{2} - x_{3} &= 4\\
-x_{1} + x_{2} &= 4\\
3 x_{1} + 3 x_{2} + 2 x_{3} &= 3\\
x_{1} - x_{3} &= 3\\
-x_{1} - x_{2} + x_{3} &= 2\\
3 x_{1} - x_{2} + x_{3} &= 1 \\
\end{align*}
\end{ejer}

{\it Soluci\'on:}
% Escribe tu solución al Ejercicio 2

% Fin del Ejercicio 2

\begin{ejer}
Estudia y resuelve si es posible el siguiente sistema de ecuaciones 
sobre el cuerpo ${\mathbb Z}_5$. (Fíjate en que la matriz de los coeficientes
es la misma que en el ejercicio anterior y puedes reutilizar la definición
de la matriz)

\begin{align*}
3 x_{2} - x_{3} &= 1\\
-x_{1} + x_{2} &= 2\\
3 x_{1} + 3 x_{2} + 2 x_{3} &= 4\\
x_{1} - x_{3} &= 2\\
-x_{1} - x_{2} + x_{3} &= 2\\
3 x_{1} - x_{2} + x_{3} &= 3 \\
\end{align*}
\end{ejer}

{\it Soluci\'on:}
% Escribe tu solución al Ejercicio 3

% Fin del Ejercicio 3

\vspace{1cm}

Cuando el sistema no es incompatible, pero al reducir no tenemos que 
todas las columnas sean pivote, tenemos los que se denominan 
{\bf sistemas compatibles indeterminados}. Para solucionarlos podemos añadir
los pivotes que nos faltan junto con parámetros libres. Veámoslo en 
un ejemplo. 

\begin{ejem} Estudia y resuelve si es posible el siguiente sistema de ecuaciones sobre $\z _7$ :
\begin{align*}
3x_{1} + x_{2} + 3x_{3} + 5x_{4} + 2x_{5} &= 3 \\
3x_{3} + 3x_{4} + x_{5} &= 1 \\
4x_{5} &= 12 \\
\end{align*}
\end{ejem}

{\it Solución:}

% Inicio del Ejemplo 5

\begin{sageblock}
A=matrix(Zmod(7),[[3,1,3,5,2],[0,0,3,3,1],[0,0,0,0,4]])
B=column_matrix(Zmod(7),[3,1,12])
AB = block_matrix([[A,B]])
R = AB.echelon_form()
\end{sageblock}

Determinamos la matriz ampliada del sistema 

\[[A|B] = \sage{AB} \]

Reducimos por filas obteniendo la matriz

\[ \sage{R} \]

En esta matriz vemos que hay tres columnas pivote y nos faltan dos pivotes. 

% Esta parte con colores sólo está puesta para aclarar el ejemplo. No la tenéis que poner vosotros.
\[
\left(\begin{array}{rrrrr|r}
\cellcolor{orange!60!white} 1 & 5 & \cellcolor{orange!30!white} 0 & 3 & \cellcolor{orange!30!white} 0 & 2 \\
\cellcolor{orange!30!white} 0 & 0 & \cellcolor{orange!60!white} 1 & 1 & \cellcolor{orange!30!white} 0 & 4 \\
\cellcolor{orange!30!white} 0 & 0 & \cellcolor{orange!30!white} 0 & 0 & \cellcolor{orange!60!white} 1 & 3
\end{array}\right)
\]

\begin{sageblock}
P.<alpha,beta> = Zmod(7)[]
M = matrix(P,5,6)
M[:3,:] = R
M[3,1] = 1
M[4,3] = 1
M[3,5] = alpha
M[4,5] = beta
M.subdivide([],5)
MR = M.echelon_form()
X = MR.subdivision(0,1)
\end{sageblock}

Añadimos los pivotes que nos faltan en la parte izquierda y ponemos parámetros
libres en la parte derecha de las filas que hemos añadido. 

% Esta parte con colores sólo está puesta para aclarar el ejemplo. No la tenéis que poner vosotros.
\[
\left(\begin{array}{rrrrr|r}
\cellcolor{orange!60!white} 1 & \cellcolor{yellow!30!white} 5 & \cellcolor{orange!30!white} 0 & \cellcolor{yellow!30!white} 3 & \cellcolor{orange!30!white} 0 & 2 \\
\cellcolor{orange!30!white} 0 & \cellcolor{yellow!30!white} 0 & \cellcolor{orange!60!white} 1 & \cellcolor{yellow!30!white} 1 & \cellcolor{orange!30!white} 0 & 4 \\
\cellcolor{orange!30!white} 0 & \cellcolor{yellow!30!white} 0 & \cellcolor{orange!30!white} 0 & \cellcolor{yellow!30!white} 0 & \cellcolor{orange!60!white} 1 & 3 \\ \hline
\cellcolor{orange!30!white} 0 & \cellcolor{yellow!60!white} 1 & \cellcolor{orange!30!white} 0 & \cellcolor{yellow!30!white} 0 & \cellcolor{orange!30!white} 0 & \alpha \\
\cellcolor{orange!30!white} 0 & \cellcolor{yellow!30!white} 0 & \cellcolor{orange!30!white} 0 & \cellcolor{yellow!60!white} 1 & \cellcolor{orange!30!white} 0 & \beta
\end{array}\right)
\]

Continuamos la reducción con esta matriz:

\[ \sage{M} \to \sage{MR} \]

Como ya tenemos la matriz identidad en la parte izquierda, la parte derecha
es la solución que buscamos
\[ X = \sage{X}. \]
donde los parámetros $\alpha,\beta$ son elementos cualesquiera de $\z_7$. Como 
en $\z_7$ hay $7$ elementos, tenemos $7$ opciones para $\alpha$ y $7$ para $\beta$
lo que nos da un total de $7^2 = 49$ soluciones. 

% Esta parte no la tenéis que escribir vosotros. El problema ya está terminado. 
\[
\begin{array}{ccccccc} 
\sage{X(alpha=0,beta=0)} & \sage{X(alpha=0,beta=1)} & \sage{X(alpha=0,beta=2)} & \sage{X(alpha=0,beta=3) }& \sage{X(alpha=0,beta=4)} & \sage{X(alpha=0,beta=5)} & \sage{X(alpha=0,beta=6)} \\ [1cm]
\sage{X(alpha=1,beta=0)} & \sage{X(alpha=1,beta=1)} & \sage{X(alpha=1,beta=2)} & \sage{X(alpha=1,beta=3) }& \sage{X(alpha=1,beta=4)} & \sage{X(alpha=1,beta=5)} & \sage{X(alpha=1,beta=6)} \\ [1cm]
\sage{X(alpha=2,beta=0)} & \sage{X(alpha=2,beta=1)} & \sage{X(alpha=2,beta=2)} & \sage{X(alpha=2,beta=3) }& \sage{X(alpha=2,beta=4)} & \sage{X(alpha=2,beta=5)} & \sage{X(alpha=2,beta=6)} \\ [1cm]
\sage{X(alpha=3,beta=0)} & \sage{X(alpha=3,beta=1)} & \sage{X(alpha=3,beta=2)} & \sage{X(alpha=3,beta=3) }& \sage{X(alpha=3,beta=4)} & \sage{X(alpha=3,beta=5)} & \sage{X(alpha=3,beta=6)} \\ [1cm]
\sage{X(alpha=4,beta=0)} & \sage{X(alpha=4,beta=1)} & \sage{X(alpha=4,beta=2)} & \sage{X(alpha=4,beta=3) }& \sage{X(alpha=4,beta=4)} & \sage{X(alpha=4,beta=5)} & \sage{X(alpha=4,beta=6)} \\ [1cm]
\sage{X(alpha=5,beta=0)} & \sage{X(alpha=5,beta=1)} & \sage{X(alpha=5,beta=2)} & \sage{X(alpha=5,beta=3) }& \sage{X(alpha=5,beta=4)} & \sage{X(alpha=5,beta=5)} & \sage{X(alpha=5,beta=6)} \\ [1cm]
\sage{X(alpha=6,beta=0)} & \sage{X(alpha=6,beta=1)} & \sage{X(alpha=6,beta=2)} & \sage{X(alpha=6,beta=3) }& \sage{X(alpha=6,beta=4)} & \sage{X(alpha=6,beta=5)} & \sage{X(alpha=6,beta=6)} \\
\end{array}
\]

% Fin del Ejemplo 5

\begin{ejer} Estudia y resuelve si es posible el siguiente sistema de ecuaciones sobre $\r $ :
\begin{align*}
x_{1} - x_{2} + x_{3} + 2 x_{4} &= 2 \\
x_{2} - 2 x_{3} - 5 x_{4} &= -1 \\
x_{2} - x_{3} - 2 x_{4} &= 0 \\
\end{align*}
\end{ejer}

{\it Soluci\'on:}

% Escribe tu solución al Ejercicio  4

% Fin del Ejercicio 4

\vspace{1cm}

En la resolución de este tipo de ecuaciones también podemos tener parámetros
que dependiendo de los valores que tengan, nos pueden dar un tipo de sistema
o de otro. Veámoslo en un ejemplo:

\begin{ejem}
Estudia y resuelve cuando sea posible el siguiente sistema de ecuaciones
sobre $\r$ dependiendo de los parámetros $a$ y $b$.

\begin{align*}
x_{1} + x_{2} + x_{3} -2 x_{5} &= a-b\\
3 x_{1} + x_{2} + 2 x_{3} -4 x_{5} &= -13\\
-3 x_{1} - x_{3} + x_{5} &= b\\
-3 x_{1} + x_{2} -2 x_{5} &= -11\\
-2 x_{1} - x_{3} + 2 x_{5} &= -a\\
\end{align*}

\end{ejem}
{\it Solución:}

% Inicio del Ejemplo 6

\begin{sageblock}
P.<a,b> = QQ[]
A =  matrix(P,[[1,1,1,0,-2],
[3,1,2,0,-4],
[-3,0,-1,0,1],
[-3,1,0,0,-2],
[-2,0,-1,0,2]])
B =  matrix(P,[[a-b],[-13],[b],[-11],[-a]])
AB = block_matrix([[A,B]])
R = AB.echelon_form()
\end{sageblock}

Construimos la matriz ampliada del sistema y la reducimos por filas
\[ \sage{AB} \to \sage{R} \]

Las dos últimas filas de esta matriz reducida nos dicen que
\[ \sage{R[3:,:5]} X = \sage{R[3:,5]}. \]
Independientemente del valor de $X$, $0X = 0$ por lo que la única posibilidad
para que este sistema tenga solución es que $\sage{R[3:,5]} = 0$. Si esto no 
sucede el sistema es incompatible. De la igualdad $\sage{R[3,5]} = 0$
deducimos que $b = 1$ y de $\sage{R[4,5]} = 0$ deducimos que $a = -6$. 
Si sustituimos el la matriz ampliada estos valores obtenemos

\begin{sageblock}
R1 = R(a=-6,b=1)
R1.subdivide(3,5)
R2 = R1[:3,:]
R2.subdivide([],5)
\end{sageblock}
 
\[ \sage{R1} \]

Las filas de ceros ya no nos dan ninguna información (sólo dicen $0=0$) y si 
las eliminamos no cambiamos las soluciones del sistema. 

\[ \sage{R2} \]

El sistema ahora tiene tres columnas pivote y nos faltan dos. Para resolverlo
añadiremos estos dos pivotes que nos faltan y en la parte derecha 
introduciremos dos parámetros libres

\begin{sageblock}
Q.<alpha,beta> = QQ[]
M = matrix(Q,5,6)
M.subdivide([],5)
M[:3,:] = R2
M[3,3] = 1
M[4,4] = 1
M[3,5] = alpha
M[4,5] = beta
MR = M.echelon_form()
\end{sageblock}
 
Reducimos la matriz y ahora obtenemos la identidad a la izquierda

\[ \sage{M} \to \sage{MR} \]

y por lo tanto ya tenemos la solución $X = \sage{MR[:,5]}$.

% Fin del Ejemplo 6

Este método de resolver sistemas de ecuaciones planteándolos como 
ecuaciones matriciales se puede generalizar también al caso en que 
la matriz de términos independientes tiene más de una columna. 

\begin{ejem}
Encuentra todas las matrices $X$ que cumplen la ecuación matricial $AX = B$
sobre el cuerpo ${\mathbb Z}_5$ siendo 
\[ A =  \left(\begin{array}{rrr}
1 & 3 & 0 \\
3 & 3 & 0 \\
1 & 3 & 2 \\
2 & 1 & 0 \\
3 & 0 & 3 \end{array}\right) \qquad 
B = \left(\begin{array}{rr}
a & 4 \\
3 & 4 \\
0 & b \\
2a & b \\
4 & a
\end{array}\right)
\]
dependiendo de los parámetros $a$ y $b$.
\end{ejem}
{\it Solución:}

% Inicio del Ejemplo 7

\begin{sageblock}
P.<a,b> = Zmod(5)[]
A = matrix(P,[[1,3,0],[3,3,0],[1,3,2],[2,1,0],[3,0,3]])
B = matrix(P,[[a,4],[3,4],[0,b],[2*a,b],[4,a]])
AB = block_matrix([[A,B]])
R = AB.echelon_form()
\end{sageblock}

Construimos la matriz ampliada del sistema y la reducimos por filas
\[ \sage{AB} \to \sage{R} \]

Las dos últimas filas de esta matriz reducida nos dicen que para que
el sistema sea compatible se tiene que cumplir
\[ b+2 = 0 \quad 3a+2 = 0 \quad a+b+1 = 0 \]
De ahí deducimos que $b = 3$ y que $a = 1$. 

\begin{sageblock}
R1 = R(a = 1, b = 3)
R1.subdivide(3,3)
R2 = R1[:3,:]
R2.subdivide(3,3)
X = R2.subdivision(0,1)
\end{sageblock}

Sustituyendo estos valores en la matriz reducida tenemos 
\[ \sage{R1} \]
y al eliminar las filas de ceros tenemos
$\sage{R2}$
y eso nos da la solución $X = \sage{X}$.

% Fin del Ejemplo 7

\begin{ejem}
Encuentra todas las matrices $X$ que cumplen la ecuación matricial $AX = B$
sobre el cuerpo $\r$ siendo 
\[ 
A = \left(\begin{array}{rrrrr}
-2 & 2 & 2 & -2 & -1 \\
0 & \frac{1}{2} & 1 & 1 & 1 \\
0 & 1 & -1 & 2 & 0
\end{array}\right) 
\qquad 
B = \left(\begin{array}{rr}
-5 & 5 \\
2 & 3 \\
-3 & -2
\end{array}\right)
\]
\end{ejem}
{\it Solución:}

% Inicio del Ejemplo 8

\begin{sageblock}
A = matrix(QQ,[[-2,2,2,-2,-1],[0,1/2,1,1,1],[0,1,-1,2,0]])
B = matrix(QQ,[[-5,5],[2,3],[-3,-2]])
AB = block_matrix([[A,B]])
R = AB.echelon_form()
\end{sageblock}

Construimos la matriz ampliada y la reducimos

\[ \sage{AB} \to \sage{R} \]

Para completar los pivotes que nos faltan debemos introducir dos
filas con los pivotes. En la parte derecha introduciremos parámetros
libres. La matriz ampliada se reduce para obtener la solución

\begin{sageblock}
P.<alpha1,alpha2,alpha3,alpha4> = QQ[]
M = matrix(P,5,7)
M[:3,:] = R
M[3,3] = 1
M[4,4] = 1
M[3,5] = alpha1
M[3,6] = alpha2
M[4,5] = alpha3
M[4,6] = alpha4
M.subdivide([],5)
MR = M.echelon_form()
X = MR.subdivision(0,1)
\end{sageblock}

\[ \sage{M} \to \sage{MR} \]

La solución es $X = \sage{X}$ siendo $\alpha_1, \alpha_2, \alpha_3, \alpha_4$
números reales cualesquiera. 

% Fin del Ejemplo 8

\begin{ejer}
Encuentra todas las matrices $X$ que cumplen la ecuación matricial $AX = B$
sobre el cuerpo ${\mathbb Z}_5$ siendo 
\[ 
A = \left(\begin{array}{rrr}
4 & 1 & 2 \\
4 & 4 & 1 \\
3 & 4 & 0 \\
0 & 4 & 2 \\
2 & 0 & 2
\end{array}\right)
\qquad 
B = \left(\begin{array}{rrr}
0 & 2 & 0 \\
1 & 3 & 3 \\
4 & 3 & 2 \\
3 & 3 & 4 \\
4 & 0 & 2
\end{array}\right)
\]
\end{ejer}
{\it Solución:}

% Escribe tu solución al Ejercicio  5

% Fin del Ejercicio 5

\newpage 

\begin{center}
{\bf Ejercicios Adicionales}
\end{center}

\begin{ejer} Estudia y resuelve si es posible el siguiente sistema de ecuaciones sobre $\z _5$:
\begin{align*}
x_{1} - x_{2} + 3 x_{4} &= 0 \\
x_{2} - x_{4} &= 3 \\
2 x_{1} + 2 x_{2} + x_{3} + 2 x_{4} &= 4 \\
2 x_{1} + 2 x_{2} + 3 x_{3} + 3 x_{4} &= 3 \\
2 x_{1} + 2 x_{2} + 3 x_{3} &= 3 \\
\end{align*}
\end{ejer}

{\it Soluci\'on:}

% Escribe tu solución al Ejercicio  6

% Fin del Ejercicio 6

\begin{ejer} Estudia y resuelve si es posible el siguiente sistema de ecuaciones 
sobre $\z_{11}$
\begin{align*}
2x_{1} - x_{2} + x_{3} + x_{4} &= 1 \\
2x_{1} + x_{} - x_{4} &= 2 \\
x_{2} + 2x_{3} + 3x_{4} &= 0 \\
\end{align*}
\end{ejer}
{\it Soluci\'on:}

% Escribe tu solución al Ejercicio  7

% Fin del Ejercicio 7

\begin{ejer} Estudia y resuelve si es posible el siguiente sistema de ecuaciones sobre $\z _7$:
\begin{align*}
x_{1} + x_{3} - x_{4} &= 2 \\
-x_{1} + x_{2} - x_{3} + x_{4} &= 2 \\
x_{2} + x_{3} - x_{4} &= 2 \\
x_{1} - x_{3} - x_{4} &= 2 \\
\end{align*}
\end{ejer}

{\it Soluci\'on:}

% Escribe tu solución al Ejercicio  8

% Fin del Ejercicio 8

\begin{ejer} Estudia y resuelve si es posible el siguiente sistema de ecuaciones sobre $\r $:
\begin{align*}
-x_{1} - 5x_{3} + x_{6} &= -1 \\
3x_{1} + x_{2} - x_{3} + x_{5} &= 0 \\
-2x_{1} + x_{4} &= 3 \\
\end{align*}
\end{ejer}

{\it Soluci\'on:}

% Escribe tu solución al Ejercicio  9

% Fin del Ejercicio 9


\end{document}
























