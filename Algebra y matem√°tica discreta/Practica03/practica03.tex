\documentclass{amsart}
\usepackage[margin = 2cm]{geometry}
\usepackage[utf8]{inputenc}
\usepackage[usefamily=sage]{pythontex}
\usepackage{colortbl} 
\usepackage{xcolor}
\usepackage[most]{tcolorbox}

\newtheorem{ejer}{Ejercicio}
\newtheorem{ejem}{Ejemplo}

\title{Práctica 3. Aritmética Entera y Modular}
\author{Escribe tu Nombre y Grupo}

\def\n{\mathbb{N}}
\def\r{\mathbb{R}}
\def\z{\mathbb{Z}}
\def\q{\mathbb{Q}}
\def\c{\mathbb{C}}

\begin{document}
\maketitle

\section{Ecuaciones Diofánticas}

Una ecuación diofántica es una ecuación en la que nos piden únicamente las
soluciones enteras. Una ecuación puede tener soluciones, por ejemplo
$$ 2x + 4y = 3 $$
tiene infinitas soluciones, pero no tiene ninguna solución entera porque
si $x$ e $y$ se sustituyen por valores enteros, el primer miembro será 
siempre un número par y no podrá ser nunca igual a $3$, por eso las técnicas
de resolución de las ecuaciones diofánticas son específicas y no nos valen
muchos de los métodos de resolución de las ecuaciones habituales.

\begin{tcolorbox}[title = Importante]
En las ecuaciones diofánticas en las que nos piden soluciones en el 
conjunto de los números enteros, las matrices y las reducciones se 
tienen que hacer en \verb|ZZ|. Hacer reducciones en \verb|QQ| en este
tipo de problemas cuando haya que hacerlas en \verb|ZZ| anula todo el
ejercicio que será valorado con $0$.
\end{tcolorbox} 

El ejemplo fundamental de ecuaciones diofánticas al que reduciremos 
muchos de los problemas del tema es el de una ecuación con dos incógnitas.
Tal y como hemos visto en teoría, este tipo de ecuaciones tiene solución
si y solo si el máximo común divisor de los coeficientes divide al término
independiente. Vamos a ver cómo resolverlos usando \verb|sage|.

\begin{ejem}
Encuentra todas las soluciones enteras de la ecuación
$$ 35 x_1 + 25 x_2 = 55 $$
\end{ejem} 

{\it Solución:}
% Inicio del Ejemplo 1

\begin{sageblock}
A = matrix(ZZ,[35,25])
At1 = block_matrix([[A.T,1]])
At1R = At1.echelon_form()
Q = At1R[:,1:].T
\end{sageblock}

Construimos la matriz de los coeficientes $A = \sage{A}$ que vamos a 
reducir por columnas, para ello la trasponemos y ampliamos con la 
matriz identidad, reducimos y obtenemos lo siguiente:
$$ \sage{At1} \to \sage{At1R}$$
El teorema fundamental de la reducción por filas, nos dice que 
$$ \sage{At1R[:,1:]} \sage{A.T} = \sage{At1R[1,:]}$$
que si trasponemos (recordemos que la traspuesta del producto es 
el producto de las traspuestas cambiadas de orden) no dice que 
$$ \sage{A} \sage{Q} = \sage{A*Q} $$
que es el equivalente por columnas al teorema fundamental de la 
reducción por filas. 

Este producto nos da dos ecuaciones:
\begin{align*}
35 \cdot 3 + 25 \cdot (-4) &= 5 \\
35 \cdot 5 + 25 \cdot (-7) &= 0 \\
\end{align*}
La primera relación es el máximo común divisor expresado como combinación
de los coeficientes. Vamos a buscar valores enteros $u_1$ y $u_2$ de forma
que multiplicando la primera ecucación por $u_1$ y la segunda por $u_2$
y sumando, obtengamos la solución del problema. 

\begin{align*}
\Big( 35 \cdot 3 + 25 \cdot (-4) &= 5 \Big) \cdot u_1 \\
\Big( 35 \cdot 5 + 25 \cdot (-7) &= 0 \Big) \cdot u_2 \\
\end{align*}

Para conseguir de esa ecuación una solución del 
sistema tenemos que multiplicar toda la ecuación por $11$ para obtener
el término independiente igual a $55$ (es decir, $u_1 = 55/5 = 11$). 
Si el término independiente no 
hubiera sido múltiplo de $5$ (el máximo común divisor de los coeficientes)
la ecuación no tendría soluciones enteras porque para cualesquiera $x_1$ y 
$x_2$ enteros, el primer miembro sería múltiplo de $5$. 

La segunda ecuación la podemos multiplicar por un parámetro libre 
$u_2 = t$ que sea un número entero y obtener las ecuaciones
\begin{align*}
35 \cdot \colorbox{orange!30!white}{$3 \cdot 11$} + 
25 \cdot \colorbox{yellow!50!white}{$(-4) \cdot 11$} &= 5 \cdot 11 = 55 \\
35 \cdot \colorbox{orange!30!white}{$5 \cdot t$} + 
25 \cdot \colorbox{yellow!50!white}{$(-7) \cdot t$} &= 0 \cdot t = 0 \\
\end{align*}

Sumando y sacando factor común los coeficientes de la ecuación tenemos
$$ 35\colorbox{orange!30!white}{$(33+5t)$} + 
25\colorbox{yellow!50!white}{$(-44-7t)$} = 55$$
con lo que obtenemos las soluciones
\begin{align*}
x_1 &= 33+5t \\
x_2 &= -44-7t
\end{align*}
siendo $t$ un número entero cualquiera.

\begin{sageblock}
Pol.<t> = ZZ[]
U = matrix([[11],[t]])
\end{sageblock}

Esto se podría también hacer automáticamente si partimos del teorema
fundamental de la reducción por columnas y tomando $X = QU$ para un vector
$U$ que representa los coeficientes por los que multiplicamos cada ecuación:

$$ \sage{A} \sage{Q} = \sage{A*Q} $$
con lo que tenemos para cualquier valor entero $t$ que 
$$ \underbrace{\sage{A}}_{A} \underbrace{\sage{Q} \sage{U}}_{X} = \sage{A*Q} \sage{U} = \underbrace{\sage{A*Q*U}}_{B}$$
y las soluciones son
$$ X = QU = \sage{Q}\sage{U} = \sage{Q*U}.$$

% Fin del Ejemplo 1

\vspace{1cm}

Resuelve el siguiente ejercicio siguiendo la muestra del ejemplo anterior.

\begin{ejer}
Encuentra todas las soluciones enteras de la ecuación
$$ 3811 x_1 + 2923 x_2 = 8621$$
\end{ejer}

{\it Solución:}

% Escribe tu solución para el Ejercicio 1

% Fin del Ejercicio 1

Lo que se hace para dos variables, también se puede hacer con más variables
añadiendo más parámetros.

\begin{ejem}
Encuentra todas las soluciones enteras de $19x_1-57x_2+38x_3 = 95$.
\end{ejem}

{\it Solución:}
% Inicio del Ejemplo 2

\begin{sageblock}
A = matrix(ZZ,[[19,-57,38]])
At1 = block_matrix([[A.T,1]])
R = At1.echelon_form()
Q = R[:,1:].T
\end{sageblock}

Construimos la matriz de los coeficientes $\sage{A}$, la trasponemos 
y ampliamos con la identidad.
$$ \sage{At1}$$
La reducimos y obtenemos la matriz de paso por la derecha
$$ \sage{R} \quad Q = \sage{Q}$$
Comprobamos el teorema fundamental de la reducción por columnas
$$ AQ = \sage{A} \sage{Q} = \sage{A*Q} $$

Esto nos da tres ecuaciones
\begin{align*}
19 \cdot 0 + (-57) \cdot 1 + 38 \cdot 2 &= 19 \\
19 \cdot 1 + (-57) \cdot 1 + 38 \cdot 1 &= 0 \\
19 \cdot 0 + (-57) \cdot 2 + 38 \cdot 3 &= 0 \\
\end{align*}

Vamos a multiplicar cada una de las ecuaciones por un valor $u_i$ y luego
los sumaremos todos para obtener la solución general. 

\begin{align*}
\Big( 19 \cdot 0 + (-57) \cdot 1 + 38 \cdot 2 &= 19 \Big) \cdot u_1 \\
\Big( 19 \cdot 1 + (-57) \cdot 1 + 38 \cdot 1 &= 0 \Big) \cdot u_2 \\
\Big( 19 \cdot 0 + (-57) \cdot 2 + 38 \cdot 3 &= 0 \Big) \cdot u_3 \\
\end{align*}

La primera ecuación nos da el máximo común divisor de los coeficientes, $19$,
que divide al término independiente $19 \cdot 5 = 95$ así que multiplicaremos
la primera ecuación por $u_1 = 5$ y la segunda y la tercera por parámetros 
libres $u_2 = s$ y $u_3 = t$ con lo que obtendremos

\begin{align*}
19 \cdot \colorbox{orange!30!white}{$0 \cdot 5$} + (-57) \cdot \colorbox{yellow!50!white}{$1 \cdot 5$} + 38 \cdot \colorbox{pink!40!white}{$2 \cdot 5$} &= 19 \cdot 5 = 95 \\
19 \cdot \colorbox{orange!30!white}{$1 \cdot s$} + (-57) \cdot \colorbox{yellow!50!white}{$1 \cdot s$} + 38 \cdot \colorbox{pink!40!white}{$1 \cdot s$} &= 0s = 0 \\
19 \cdot \colorbox{orange!30!white}{$0 \cdot t$} + (-57) \cdot \colorbox{yellow!50!white}{$2 \cdot t$} + 38 \cdot \colorbox{pink!40!white}{$3 \cdot t$} &= 0t = 0 \\
\end{align*}

Sumando y sacando factor común tenemos 

$$ 19 \colorbox{orange!30!white}{$s$} -57 \colorbox{yellow!50!white}{$(5+s+2t)$} + 38 \colorbox{pink!40!white}{$(10+s+3t)$} = 95$$

y de ahí sacamos el valor de las variables

\begin{align*}
x_1 &= s \\
x_2 &= 5+s+2t \\
x_3 &= 10+s+3t \\
\end{align*}


\begin{sageblock}
Pol.<s,t> = ZZ[]
U = matrix([[5],[s],[t]])
\end{sageblock}

También se puede hacer automáticamente partiendo de la relación fundamental 
de la reducción por columnas
$$ AQ = \sage{A} \sage{Q} = \sage{A*Q} $$
y multiplicando por $U = \sage{U}$ (para parámetros libres $s$ y $t$ en el conjunto de los números enteros) tenemos
$$ AX = \underbrace{\sage{A}}_A \underbrace{\sage{Q} \sage{U}}_X = \sage{A*Q} \sage{U} = \sage{A*Q*U} = B$$
y por lo tanto tenemos las soluciones
$$ X = QU = \sage{Q*U}.$$

% Fin del Ejemplo 2

\begin{ejer}
Calcula todas las soluciones enteras de la ecuación diofántica
$$ 741 x_1 + 1311 x_2 + 897 x_3 = 363$$
\end{ejer}

{\it Solución:}
% Escribe tu Solución para el Ejercicio 2

% Fin del Ejercicio 2

\section{Teorema Chino de los Restos}

El problema básico que resuelve el Teorema Chino de los Restos es encontrar los
valores de $n$ tales que se cumplan simultáneamente las relaciones

\begin{align*}
n &\equiv a_1 \pmod{m_1} \\
n &\equiv a_2 \pmod{m_2} \\
\end{align*}

Cuando este problema tiene solución, será de la forma

$$ n \equiv a \pmod{mcm(m_1,m_2)} $$

Este tipo de problemas los podemos reducir a una ecuación diofántica lineal. 
Vamos a verlo con un ejemplo.

\begin{ejem}
\label{EjemDiof2}
Encuentra todos los números enteros $n$ tales que 
\begin{align*}
n &\equiv 12 \pmod{35} \\
n &\equiv 42 \pmod{55} \\
\end{align*}
\end{ejem}

{\it Solución:}
% Inicio del Ejemplo 3

La primera relación nos dice que $n = 12+35x_1$ y la segunda $n = 42+55x_2$
para valores enteror indeterminados $x_1$ y $x_2$. Igualando ambas ecuaciones
tenemos que $$ 12+35x_1 = 42+55x_2$$
Esto nos da la ecuación diofántica
$$ 35x_1 - 55 x_2 = 42-12 = 30 $$

\begin{sageblock}
A = matrix(ZZ,[[35,-55]])
At1 = block_matrix([[A.T,1]])
At1R = At1.echelon_form()
Q = At1R[:,1:].T
\end{sageblock}

Tomamos la matriz de los coeficientes $A = \sage{A}$ y vamos a reducir por
columnas, para ello trasponemos y ampliamos con la identidad para reducir por
filas
$$ \sage{At1} \to \sage{At1R}$$
La matriz que queda a la derecha es la traspuesta de la matriz de paso por la
derecha, que podemos comprobar que cumple el teorema fundamental de la 
reducción por columnas
$$ Q = \sage{Q} \qquad \sage{A} \sage{Q} = \sage{A*Q}$$
El máximo común divisor es $5$ que divide al término independiente que es $30$.
Para conseguirlo tenemos que multiplicar por $6$. 

\begin{sageblock}
Pol.<t> = ZZ[]
U = matrix(Pol,[[6],[t]])
X = Q*U
\end{sageblock}

La solución es $X = \sage{Q} \sage{U} = \sage{X}$ para un valor entero
cualquiera $t$ y que podemos comprobar
viendo que $$AX = \sage{A} \sage{X} = \sage{A*X}.$$

Con los valores $x_1$ y $x_2$ podemos obtener el valor de $n$. Cualquiera
de las dos relaciones nos sirve
$$ n = 12+35x_1 = 12+35(\sage{X[0,0]}) = \sage{12+35*X[0,0]} $$
$$ n = 42+55x_2 = 42+55(\sage{X[1,0]}) = \sage{42+55*X[1,0]} $$

Esto es equivalente a decir que $n$ es $1692$ salvo múltiplos de $385$, es
decir
$$ n \equiv 1692 \pmod{385}$$
Podemos reducir módulo $385$ para obtener un valor de $n$ entre $0$ y $384$,
$$ n \equiv 1692 \equiv \sage{Zmod(385)(1692)} \pmod{385} $$
Podemos ver que el resultado nos ha salido módulo $385 = \sage{factor(385)}$
que es el mínimo común múltiplo de $35 = \sage{factor(35)}$ y $55 = 
\sage{factor(55)}$.

Para entender un poco mejor lo que está pasando en este problema, vamos
a replantearlo del siguiente modo: Llamemos $n = x_0$ y vamos a escribir
de nuevo las ecuaciones

\begin{align*}
x_0 &= 12+35x_1 \\
x_0 &= 42+55x_2 \\
\end{align*}
o lo que es lo mismo
\begin{align*}
x_0 - 35x_1 &= 12 \\
x_0 - 55 x_2 &= 42 \\
\end{align*}
Lo que hacemos al igualar los valores de $n$ en realidad es resolver 
este sistema de ecuaciones diofánticas restando a la segunda ecuación
la primera para así eliminar $x_0$ (que es $n$) lo que nos deja
\begin{align*}
x_0 - 35x_1 &= 12 \\
35x_1 - 55 x_2 &= 30 \\
\end{align*}

Resolvemos la segunda ecuación en dos variables y luego utilizamos la 
primera para sacar el valor de $x_0$ que es el que necesitamos. Por lo
tanto lo que estamos haciendo con este sistema no es más que resolver 
un caso particular de sistema de ecuaciones diofánticas de dos ecuaciones
y tres incógnitas. 

% Fin del Ejemplo 3

\vspace{1cm}

\begin{ejer}
Calcula todos los valores enteros $n$ tales que 
\begin{align*}
n &\equiv 20 \pmod{111} \\
n &\equiv 56 \pmod{75} 
\end{align*}
\end{ejer}

{\it Solución: }

% Escribe tu solución para el Ejercicio 3

% Fin del Ejercicio 3

\vspace{1cm}

Cuando en lugar de tener dos relaciones de congruencia, tenemos tres o más, lo
más eficiente es ir agrupando de dos en dos las relaciones para ir reduciendo su
número ya que la combinación de dos relaciones de congruencia es una nueva 
relación de congruencia. Veámoslo con un ejemplo:

\begin{ejem}
Encuentra todos los valores enteros $n$ tales que 
\begin{align*}
n &\equiv 27 \pmod{102} \\
n &\equiv 12 \pmod{35} \\
n &\equiv 42 \pmod{55} \\
\end{align*}
\end{ejem}


% Inicio del Ejemplo 4
{\it Solución 1: }

Lo primero que vamos a notar es que en el Ejemplo~\ref{EjemDiof2} encontramos 
los valores $n$ que cumplían las dos últimas ecuaciones, por lo tanto las dos 
relaciones de congruencia

\begin{align*}
n &\equiv 12 \pmod{35} \\
n &\equiv 42 \pmod{55} \\
\end{align*}

son equivalentes a la relación de congruencia 

\begin{align*}
n &\equiv 152 \pmod{385} \\
\end{align*}

y por lo tanto nuestro problema lo podemos reducir a encontrar los valores $n$ tales
que 
\begin{align*}
n &\equiv 27 \pmod{102} \\
n &\equiv 152 \pmod{385} \\
\end{align*}
que resolveremos usando la técnica habitual de dos variables:
$$ n = 27+102x_1 \qquad n = 152+385x_2 $$
igualando las dos ecuaciones tenemos que $$27+102x_1 = 152+385x_2 $$
de donde obtenemos la ecuación en la forma habitual $$ 102x_1 - 385 x_2 = 152-27 = \sage{152-27}$$

\begin{sageblock}
A = matrix(ZZ,[[102,-385]])
At1 = block_matrix([[A.T,1]])
At1R = copy(At1.echelon_form())
At1R.subdivide([],[1])
Q = At1R.subdivision(0,1).T
Pol.<t,x1,x2> = ZZ[]
U = matrix(Pol,[[125],[t]])
XX = Q*U
X = matrix(Pol,[[x1],[x2]])
sol = 27+102*XX[0,0]
\end{sageblock}

Tomamos la matriz de los coeficientes $A = \sage{A}$ la trasponemos, ampliamos con la 
identidad y reducimos sobre el conjunto de los números enteros
$$ \sage{At1} \to \sage{At1R}$$
La matriz de paso por la derecha es $Q = \sage{Q}$ y el teorema fundamental de la reducción
por columnas nos dice que 
$$ \sage{A} \sage{Q} = \sage{A*Q}$$
En este caso el máximo común divisor de los coeficientes es $1$ y la ecuación seguro que tiene
solución que podremos obtener tomando 
$$ \sage{A} \sage{Q} \sage{U} = \sage{A} \sage{XX} = \sage{A*XX}$$
lo que nos dice que las soluciones para $x_1$ y $x_2$ son 
$$ X = \sage{X} = \sage{XX}$$
De ahí, sustituyendo en el valor de $n$ tenemos que 
$$ n = 27+102x_1 = 27+102(\sage{XX[0,0]}) = \sage{sol}$$
El resultado nos lo da salvo múltiplos de 
$\sage{sol.coefficients()[0]} = \sage{factor(sol.coefficients()[0])}$ 
que es precisamente el $mcm(102,35,55) = 
mcm(\sage{factor(102)},\sage{factor(35)},\sage{factor(55)})$. Podemos reducir y obtener 
$$ n = \sage{sol.coefficients()[1]} \equiv 
\sage{Zmod(sol.coefficients()[0])(sol.coefficients()[1])} 
\pmod{\sage{sol.coefficients()[0]}}.$$

{\it Solución 2: }

\begin{tcolorbox}[title = Nota del Coordinador]
Esta solución está puesta para que podáis ver un método alternativo, pero
sólo necesitáis aprender a resolver este problema usando la primera solución que,
aunque es más larga que la segunda (recordad que a la que está aquí escrita hay 
que añadir la solución del Ejemplo~\ref{EjemDiof2}) reduce un problema complejo
a dos problemas básicos pueden ser largos, pero son sencillos. 

\flushright Leandro Marín
\end{tcolorbox} 


Vamos a ver una segunda solución a este problema planteándolo como sistema de 
ecuaciones diofánticas. Vamos a llamar $n = x_0$ y reescribiremos las relaciones
del enunciado como 
\begin{align*}
x_0 &= 27 + 102 x_1 \\
x_0 &= 12 + 35 x_2 \\
x_0 &= 42 + 55 x_3 \\ 
\end{align*}

\begin{sageblock}
A = matrix(ZZ,[[1,-102,0,0],[1,0,-35,0],[1,0,0,-55]])
B = matrix(ZZ,[[27],[12],[42]])
Pol.<x0,x1,x2,x3,u0,u1,u2,u3,t> = ZZ[]
X = matrix(Pol,[[x0],[x1],[x2],[x3]])
AB = block_matrix([[A,B]])
ABR = copy(AB.echelon_form())
ABR.subdivide([],4)
AA = ABR.subdivision(0,0)
BB = ABR.subdivision(0,1)
\end{sageblock}

Escrito en forma matricial, estas relaciones son equivalentes a la ecuación
matricial
$$ \sage{A} \sage{X} = \sage{B}$$
Construimos la ampliada del sistema y reducimos sobre el conjunto de los números
enteros
$$ \sage{AB} \to \sage{ABR}$$
El sistema diofántico que tenemos que resolver en su forma reducida es 
$$ \sage{AA} \sage{X} = \sage{BB}$$

\begin{sageblock}
AAt1 = block_matrix([[AA.T,1]])
AAt1R = copy(AAt1.echelon_form())
AAt1R.subdivide([],3)
Q = AAt1R.subdivision(0,1).T
U = matrix(Pol,[[u0],[u1],[u2],[u3]])
\end{sageblock}

Para reolverlo vamos a hacer lo mismo que cuando tenemos una sola ecuación, vamos
a sacar la matriz de paso por la derecha con una reducción por columnas. Para ello
trasponemos y reducimos por filas

$$ \sage{AAt1} \to \sage{AAt1R} $$

La matriz de paso por la derecha es 
$$ Q = \sage{Q}$$
y el teorema fundametal de la reducción por columnas
nos dice que 
$$\sage{AA} \sage{Q} = \sage{AA*Q} = R'$$

Nuestro objetivo es encontar un vector $U$ tal que $X = QU$ sea la solución del 
sistema. Vamos a ver qué condiciones tendría que cumplir

$$ \sage{BB} = \sage{AA} \sage{X} = $$
$$ \sage{AA} \overbrace{\left(\sage{Q} \sage{U}\right)}^X = $$
$$ \underbrace{\left(\sage{AA} \sage{Q} \right)}_{R'} \sage{U} = $$
$$ \sage{AA*Q} \sage{U} = \sage{AA*Q*U}$$

Tomando el principio y el final tenemos que  $\sage{AA*Q*U} = \sage{BB}$
y de ahí deducimos que $u_0 = 42$, $u_1 = 15$, $u_2 = 6$ (la división en este caso
es posible, si no hubiera sido posible el sistema no tendría solución) y $u_3$
puede tomar cualquier valor, por lo que la llamaremos $t$ como solemos hacer a los
parámetros libres. Esto nos dice que 
$$ X = \sage{Q} \sage{U(u0=42,u1=15,u2=6,u3=t)} = \sage{Q*U(u0=42,u1=15,u2=6,u3=t)}$$

El valor que realmente nos interesa es el de $x_0 = n$ que nos dice que es
$$ n = 39270t + 511047$$
para un valor cualquiera de $t$, o lo que es lo mismo 
$$ n \equiv 511047 \pmod{39270}$$
Si reducimos para obtener el representante más pequeño, tenemos que 
$$ n \equiv 511047  \equiv \sage{Zmod(39270)(511047)} \pmod{39270}$$
que es el mismo valor que nos salía por el primer método.
% Fin del Ejemplo 4


\begin{ejer}
Encuentra todos los valores enteros $n$ tales que
\begin{align*}
n &\equiv 18 \pmod{34} \\
n &\equiv 24 \pmod{25} \\
n &\equiv 8 \pmod{28} \\
\end{align*}
\end{ejer}

{\it Solución: }

% Escribe tu solución para el Ejercicio 4

% Fin del Ejercicio 4

\section{Despejando $n$ en las congruencias.}

En los problemas de Teorema Chino de los Restos necesitamos ecuaciones del tipo 
$$ n \equiv a \pmod{m} $$
o lo que es lo mismo, $n = a$ en ${\mathbb Z}_m$. 
A veces no tenemos las ecuaciones de esta forma y tenemos expresiones
un poco más complicadas del tipo 
$$ 5n+4 \equiv 6 \pmod{7}$$
Si el módulo es primo como en este ejemplo, ésta ecuación se puede resolver
despejando $5n = 6-4$ y por lo tanto $n = 5^{-1} (6-4) = 3\cdot 2 = 6$ en 
${\mathbb Z}_7$. El inverso modular $5^{-1}$ se tiene que calcular en 
${\mathbb Z}_7$. Esto se puede hacer aunque el módulo no sea primo 
siempre que el inveso modular exista. Por ejemplo

\begin{ejem}
Despeja el valor de $n$ en la siguiente relación de congruencia
$$ 128n + 65 \equiv 41 \pmod{4477}. $$
\end{ejem}
{\it Solución: }

% Inicio del Ejemplo 5

El $65$ no da problema porque lo podemos pasar restando al segundo miembro.
$$ 128 n \equiv 41 - 65 = -24 \equiv 4553 \pmod{4477}$$
Para eliminar el $128$ debemos comprobar si ese elemento es invertible 
módulo $4477$ lo cual sucederá cuando el máximo común divisor entre $128$ y 
$4477$ sea $1$. Podríamos hacerlo con un único comando en \verb|sage| pero 
lo vamos a hacer usando el algoritmo de euclides extendido.

\begin{sageblock}
A = matrix(ZZ,[[218,4477]])
At1 = block_matrix([[A.T,1]])
At1R = copy(At1.echelon_form())
At1R.subdivide([],[1])
Q = At1R.subdivision(0,1).T
\end{sageblock}

Procedemos como en el caso de las ecuaciones diofánticas. 

$$\sage{At1} \to \sage{At1R}$$
Calculamos la matriz de paso por la derecha $Q$ y el teorema fundamental 
de la reducción por columnas nos dice que 
$$Q = \sage{Q} \qquad \sage{A} \sage{Q} = \sage{A*Q}$$

El máximo común divisor es $1$ y el Lema de Bézout nos dice que 
$$ 1 = 218 \cdot 3635 + 4477 \cdot (-177) $$
por lo que $3635$ es el inverso modular de $218$ módulo $4477$.

% Fin del Ejemplo 5

También podemos tener el caso en que el coeficiente que multiplica a $n$
no sea invertible. En ese caso procederemos del siguiente modo:

\begin{ejem}
Despeja el valor de $n$ en la siguiente relación de congruencia:
$$ 165 n \equiv 3618 \pmod{4551} $$
\end{ejem}

{\it Solución: }

% Inicio del Ejemplo 6

Esta relación es equivalente a decir que $165n = 3618+4551t$ para algún
valor de $t$. Vamos a ver si $165$ es invertible módulo $4551$ calculando
el máximo común divisor de $165$ y $4551$.

\begin{sageblock}
A = matrix(ZZ,[[165,4551]])
R = A.T.echelon_form()
\end{sageblock}

Tomamos la matriz columna con ambos valores y reducimos:

$$ \sage{A.T} \to \sage{R}$$

Esto nos dice que el máximo común divisor es $3$ y vamos a dividir toda
la ecuación $165n = 3618+4551t$ por $3$.
$$ \sage{165/3} n = \sage{3618/3} + \sage{4551/3} t$$
Todos los términos han resultado ser divisibles por $3$ (para $165$ y 
$4551$ estábamos seguros porque $3$ era su máximo común divisor, pero
podría haber fallado para el término independiente $3618$ en cuyo caso
no habría ningún $n$ cumpliendo las condiciones). 
La ecuación $55n = 1206+1517t$ para algún valor de $t$ es equivalente
a
$$ 55n \equiv 1206 \pmod{1517}$$
con la ventaja de que ahora el máximo común divisor de $55$ y $1517$ sí
es $1$ y por lo tanto podemos calcular el inverso modular de $55$ módulo
$1517$ y proceder como en el ejemplo anterior, o directamente poner
$$ n \equiv 55^{-1} \cdot 1206 \equiv 
\sage{Zmod(1517)(55)^-1} \cdot 1206
\equiv \sage{Zmod(1517)(55)^-1 * 1206} \pmod{1517}.$$

% Fin del Ejemplo 6

\begin{ejer}
Despeja del valor de $n$ en las siguientes relaciones de congruencia:
\begin{enumerate}
\item $5n + 1 \equiv 4 \pmod{7}$
\item $45n - 23 \equiv 44 \pmod{121}$
\item $15n - 4 \equiv 6 \pmod{20}$
\item $814 n +14 \equiv 4343 \pmod{4551}$
\end{enumerate}
\end{ejer}
{\it Solución: }

\begin{enumerate}
\item $5n + 1 \equiv 4 \pmod{7}$
% Escribe tu solución para el Ejercicio 5, apartado 1

% Fin del apartado 1
\item $45n - 23 \equiv 44 \pmod{121}$
% Escribe tu solución para el Ejercicio 5, apartado 2

% Fin del apartado 2
\item $15n - 4 \equiv 6 \pmod{20}$
% Escribe tu solución para el Ejercicio 5, apartado 3

% Fin del apartado 3
\item $814 n +14 \equiv 4343 \pmod{4551}$
% Escribe tu solución para el Ejercicio 5, apartado 4

% Fin del apartado 4
\end{enumerate}

\section{Ejercicios Adicionales}

\begin{ejer}[Examen Junio 2024]
En nuestro almacén recibimos furgonetas pequeñas con una capacidad de $108$ ordenadores y 
salen furgonetas grandes con una capacidad de $201$ ordenadores. En este momento 
tenemos en el almacén $74$ ordenadores, pero sospechamos que podríamos haber sufrido
un robo de un ordenador. Sabiendo que las furgonetas siempre van al máximo de su 
capacidad determinia si efectivamente hemos sufrido ese robo y cuantas furgonetas 
han podido entrar y salir del almacén.
\end{ejer}

{\it Solución: }

% Escribe tu solución para el Ejercicio 6

% Fin del Ejercicio 6

\end{document}


